\thispagestyle{thachthuctoanhocnone}
\pagestyle{thachthuctoanhoc}
\everymath{\color{thachthuctoanhoc}}
\graphicspath{{../thachthuctoanhoc/pic/}}
\begingroup
\AddToShipoutPicture*{\put(0,616){\includegraphics[width=19.3cm]{../thachthuctoanhoc/bannerthachthuc}}}
\centering
\vspace*{4cm}
\endgroup
\vspace*{-8pt}
\begin{tBox}
	\begin{itemize}[leftmargin = 13pt, itemsep = 1.0pt] 
		\item Mỗi bài toán đề xuất (kèm theo lời giải) cần được nêu rõ là bài sáng tác hay bài sưu tầm.
		%		\item Mỗi bài toán đề xuất (kèm theo lời giải) cần được nêu rõ là bài sáng tác hay bài sưu tầm (nếu là bài sưu tầm, cần ghi rõ nguồn).
		\item Bài giải cho mỗi bài toán cần được trình bày trong một file riêng hoặc
		một tờ giấy riêng.
		\item  Người đề xuất bài toán hoặc gửi bài giải cho các bài toán trong mục ``Thách thức kỳ này" cần ghi rõ họ, đệm, tên và nơi làm việc/học tập, số điện thoại liên hệ. Nếu là học sinh (hoặc sinh viên) cần ghi rõ là học sinh lớp mấy (hoặc sinh viên năm thứ mấy).
		\item Các bài toán trong mục Thách thức kỳ này hướng tới các độc giả là học sinh phổ thông; được phân chia thành các mức độ $B$, $A$, và được sắp xếp theo độ khó tăng dần, theo đánh giá chủ quan của Ban biên tập. Các bài toán mức độ $B$ không đòi hỏi các kiến thức vượt quá chương trình môn Toán cấp THCS; các bài toán mức độ $A$ không đòi hỏi các kiến thức vượt quá chương trình môn Toán cấp THPT.
		\item Cách thức gửi bài toán đề xuất hoặc lời giải: gửi file thu được bằng cách scan, ảnh chụp (rõ nét) của bản viết tay, hoặc được soạn thảo bằng các phần mềm Latex, Word tới \url{bbt@pi.edu.vn} hoặc gửi qua đường bưu điện tới Tòa soạn (xem địa chỉ tại bìa $2$).
		\item Hạn gửi lời giải cho các bài toán P$791$--P$800$: trước ngày $15/4/2024$.
	\end{itemize}
\end{tBox}
\begin{center}
	\vspace*{-5pt}
	\textbf{\color{thachthuctoanhoc}\color{thachthuctoanhoc}\color{thachthuctoanhoc}\color{thachthuctoanhoc}\color{thachthuctoanhoc}THÁCH THỨC KỲ NÀY}
	\vspace*{-5pt}
\end{center}
\begin{multicols}{2}
	\setlength{\abovedisplayskip}{4pt}
	\setlength{\belowdisplayskip}{4pt}
	{\color{thachthuctoanhoc}{\usefont{T5}{qag}{b}{n} P791.}}
	(Mức $B$) Trong một nhóm $21$ thành phố, mỗi thành phố đều có đều có tuyến bay hai chiều tới ít nhất $3$ thành phố khác. Chứng minh rằng có một thành phố mà từ đó có thể bay tới $4$ thành phố khác. 
	\begin{flushright}
		\textit{Bằng Linh, Phú Thọ (st)}
	\end{flushright}
	{\color{thachthuctoanhoc}{\usefont{T5}{qag}{b}{n} P792.}}
	(Mức $B$) Tìm chữ số đầu tiên khác không, tính từ phải sang trái, trong biểu diễn thập phân của số $A=30!$.
	\begin{flushright}
		\textit{Nguyễn Phùng Đức Hiệp, Hà Nội }
	\end{flushright}
	{\color{thachthuctoanhoc}{\usefont{T5}{qag}{b}{n} P793.}}
	(Mức $B$) Tìm tất cả các  số nguyên $a,b$ sao cho $(ax+by)^2$ chia hết cho $x^2+y^2$, với mọi số nguyên dương $x,y$.
	\begin{flushright}
		\textit{Tô Trung Hiếu, Nghệ An}
	\end{flushright}
	
	{\color{thachthuctoanhoc}{\usefont{T5}{qag}{b}{n} P794.}}
	(Mức $B$) Xét các số nguyên dương $a,b,c$ thỏa mãn $2a+3b+5c=100$. Tìm giá trị lớn nhất và nhỏ nhất của biểu thức $P=abc$.
	\begin{flushright}
		\textit{Hoàng Ngọc Minh, Hà Nội}
	\end{flushright}
	{\color{thachthuctoanhoc}{\usefont{T5}{qag}{b}{n} P795.}}
	(Mức $B$) Hai bạn Vinh và Tùng mỗi bạn có $n$ tấm thẻ, trên mỗi tấm thẻ ghi các số nguyên dương từ 1 đến $n$ (mỗi thẻ ghi đúng $1$ số). Bạn Hưng có $n$ cái hộp trống. Hưng bảo Vinh và Tùng lần lượt bỏ ngẫu nhiên vào mỗi hộp một thẻ của mình. Sau đó, Hưng ghi lên mỗi hộp giá trị tuyệt đối của hiệu hai số trong hộp. Tùng khẳng định: Chắc chắn có hai hộp sẽ được Hưng ghi cùng một số.  Hỏi bạn Tùng nói đúng hay sai trong các tình huống dưới đây.
	\vskip 0.05cm
	$a)$ $n=99.$
	\vskip 0.05cm
	$b)$ $n=100.$
	\begin{flushright}
		\textit{Nguyễn Đình Tùng, Hà Nội}
	\end{flushright}
	{\color{thachthuctoanhoc}{\usefont{T5}{qag}{b}{n} P796.}}
	(Mức $B$) Cho hai hình vuông $XYZT, PQRS$ có cạnh bằng nhau.  Các cạnh của chúng cắt nhau tạo thành một bát giác lồi $ABCDEFGH$ như hình vẽ. Chứng minh rằng  
	\begin{align*}
		AB \!+\! CD \!+\! EF \!+\! GH \!=\! BC \!+\! DE \!+\! FG \!+\! HA.
	\end{align*}
	\begin{figure}[H]
		\vspace*{-5pt}
		\centering
		\captionsetup{labelformat= empty, justification=centering}
		\begin{tikzpicture}[thachthuctoanhoc,scale=0.9]
			\draw  (0.,4.)-- (0.,0.);
			\draw  (0.,0.)-- (4.,0.);
			\draw  (4.,0.)-- (4.,4.);
			\draw  (4.,4.)-- (0.,4.);
			\draw  (1.42,4.58)-- (-0.8078121195006047,1.2578240323236605);
			\draw  (1.42,4.58)-- (-0.8078121195006047,1.2578240323236605);
			\draw  (-0.8078121195006047,1.2578240323236605)-- (2.514363848175734,-0.9699880871769437);
			\draw  (2.514363848175734,-0.9699880871769437)-- (4.742175967676339,2.352187880499395);
			\draw  (4.742175967676339,2.352187880499395)-- (1.42,4.58);
			\draw [fill=white] (0.,4.) circle (1.5pt);
			\draw (-0.06727272727272707,4.4018181818181805) node {$S$};
			\draw [fill=white] (0.,0.) circle (1.5pt);
			\draw (-0.15818181818181798,-0.2709090909090915) node {$R$};
			\draw [fill=white] (4.,0.) circle (1.5pt);
			\draw (4.26,-0.14363636363636423) node {$Q$};
			\draw [fill=white] (4.,4.) circle (1.5pt);
			\draw (4.132727272727273,4.329090909090908) node {$P$};
			\draw [fill=white] (1.42,4.58) circle (1.5pt);
			\draw (1.550909090909091,4.91090909090909) node {$X$};
			\draw [fill=white] (-0.8078121195006047,1.2578240323236605) circle (1.5pt);
			\draw (-0.9909090909088,1.02) node {$T$};
			\draw [fill=white] (2.514363848175734,-0.9699880871769437) circle (1.5pt);
			\draw (2.932727272727273,-0.9072727272727279) node {$Z$};
			\draw [fill=white] (4.742175967676339,2.352187880499395) circle (1.5pt);
			\draw (4.878181818181818,2.6927272727272724) node {$Y$};
			\draw [fill=white] (1.0310588235294116,4.) circle (1.5pt);
			\draw (0.9145454545454548,4.292727272727272) node {$A$};
			\draw [fill=white] (2.2849122807017546,4.) circle (1.5pt);
			\draw (2.4054545454545457,4.292727272727272) node {$B$};
			\draw [fill=white] (4.,2.849882352941175) circle (1.5pt);
			\draw (4.32727272727273,3.1472727272727266) node {$C$};
			\draw [fill=white] (4.,1.2454342444908202) circle (1.5pt);
			\draw (4.241818181818182,1.2018181818181812) node {$D$};
			\draw [fill=white] (3.164826447812038,0.) circle (1.5pt);
			\draw (3.423636363636364,-0.25272727272727336) node {$E$};
			\draw [fill=white] (1.0678903848416958,0.) circle (1.5pt);
			\draw (1.0054545454545456,-0.25272727272727336) node {$F$};
			\draw [fill=white] (0.,0.716114728658549) circle (1.5pt);
			\draw (-0.30363636363636337,0.6563636363636357) node {$G$};
			\draw [fill=white] (0.,2.4624561403508776) circle (1.5pt);
			\draw (-0.35818181818181793,2.674545454545454) node {$H$};
		\end{tikzpicture}
		\vspace*{-10pt}
	\end{figure}
	\begin{flushright}
		\textit{Nguyễn Thúy Quỳnh, Hà Nội (st)}
	\end{flushright}
	{\color{thachthuctoanhoc}{\usefont{T5}{qag}{b}{n} P797.}}
	(Mức $A$) Có bao nhiêu bộ sắp thức tự các số nguyên dương $(a,b,c)$ thỏa mãn $abc=2023^{2024}$.
	\begin{flushright}
		\textit{Vũ Hồng Sơn, Phú Thọ}
	\end{flushright}
	{\color{thachthuctoanhoc}{\usefont{T5}{qag}{b}{n} P798.}}
	(Mức $A$) Cho $f(x)$ là một đa thức hệ số thực có bậc là một số nguyên dương $n \ge 2$. Giả sử rằng, $f(x)$ có $n$ nghiệm thực là $ \alpha_{1},\cdots,\alpha_{n}$ và $f^\prime(x)$ có $n-1$ nghiệm thực là $\alpha_1^\prime,\ldots,\alpha_{n-1}^\prime$ (không nhất thiết phân biệt).  Chứng minh rằng
	\begin{align*}
		\prod_{\substack{1\le k,\ell \le n \\ k \ne \ell}}(\alpha_{k}-\alpha_{\ell})=n^{n}\prod_{\substack{1\le i \le n\\ 1\le j \le n-1}}(\alpha_{i}-\alpha'_{j}).
	\end{align*}
	(Chú ý: Các nghiệm của $f(x)$ và $f^\prime(x)$ không nhất thiết phân biệt).
	\begin{flushright}
		\textit{Dương Hồng Sơn, England }
	\end{flushright}
	{\color{thachthuctoanhoc}{\usefont{T5}{qag}{b}{n} P799.}}
	(Mức $A$) Tìm tất cả các số nguyên tố $p$ thoả mãn
	\begin{align*}
		\sum_{i=1}^{\left[\frac p2\right]} i \cdot C_p^i \equiv 0\pmod{p^2}.
	\end{align*}
	\begin{flushright}
		\textit{Trần Minh Hiền, Bình Phước}
	\end{flushright}
	{\color{thachthuctoanhoc}{\usefont{T5}{qag}{b}{n} P800.}}
	(Mức $A$) Cho tứ giác lồi $ABCD$ có $AB + CD > BC + DA$. Các đường chéo $AC$, $BD$ cắt nhau tại $O$. Gọi $x, y, z, t$ lần lượt là khoảng cách từ điểm $O$ đến các đường thẳng $AB, BC, CD, DA$. Chứng minh rằng 
	\begin{align*}
		\dfrac{1}{x} + \dfrac{1}{z} > \dfrac{1}{y} + \frac{1}{t}.
	\end{align*}
	\definecolor{ffqqqq}{rgb}{1,0,0}
	\definecolor{qqzzcc}{rgb}{0,0.6,0.8}
	\definecolor{qqqqff}{rgb}{0,0,1}
	\definecolor{qqqqffa}{rgb}{1,1,1}
	\begin{tikzpicture}[thachthuctoanhoc,scale=0.78]
		\draw[color=ffqqqq] (-1.933047864809872,1.6169496023416794) -- (-1.6595434766014723,1.6890283761294527) -- (-1.7316222503892453,1.9625327643378525) -- (-2.005126638597645,1.8904539905500792) -- cycle; 
		\draw[color=ffqqqq] (1.808456113111784,1.4401828355632256) -- (1.9130003775568354,1.177370170777749) -- (2.1758130423423117,1.2819144352228005) -- (2.0712687778972603,1.544727100008277) -- cycle; 
		\draw[color=ffqqqq] (-1.5358683620315439,-2.317157287525381) -- (-1.818711074506163,-2.317157287525381) -- (-1.818711074506163,-2.6) -- (-1.5358683620315439,-2.6) -- cycle; 
		\draw[color=ffqqqq] (-3.8368552706185506,0.4807837728633495) -- (-3.791839649886504,0.7600212952168265) -- (-4.071077172239981,0.8050369159488732) -- (-4.116092792972028,0.5257993935953962) -- cycle; 
		\draw  (-4.62,-2.6)-- (3.72,-2.6);
		\draw  (3.72,-2.6)-- (1.56,2.83);
		\draw  (1.56,2.83)-- (-3.98,1.37);
		\draw  (-3.98,1.37)-- (-4.62,-2.6);
		\draw  (-3.98,1.37)-- (3.72,-2.6);
		\draw  (-4.62,-2.6)-- (1.56,2.83);
		\draw [dashed,color=qqzzcc] (-2.005126638597645,1.8904539905500792)-- (-1.5358683620315436,0.10984381782665315);
		\draw [dashed,color=qqzzcc] (-1.5358683620315436,0.10984381782665315)-- (2.0712687778972603,1.544727100008277);
		\draw [dashed,color=qqzzcc] (-1.5358683620315436,0.10984381782665315)-- (-1.5358683620315439,-2.6);
		\draw [dashed,color=qqzzcc] (-4.116092792972028,0.5257993935953962)-- (-1.5358683620315436,0.10984381782665315);
		\draw (-1.84,1.55) node[anchor=north west] {$x$};
		\draw (0.58,1.05) node[anchor=north west] {$y$};
		\draw (-2,-1.01) node[anchor=north west] {$z$};
		\draw (-3.56,0.4) node[anchor=north west] {$t$};
		\draw [fill=white] (-3.98,1.37) circle (1.5pt);
		\draw[color=qqqqff] (-4.42,1.94) node {$A$};
		\draw [fill=white] (1.56,2.83) circle (1.5pt);
		\draw[color=qqqqff] (1.72,3.14) node {$B$};
		\draw [fill=white] (3.72,-2.6) circle (1.5pt);
		\draw[color=qqqqff] (3.64,-2.98) node {$C$};
		\draw [fill=white] (-4.62,-2.6) circle (1.5pt);
		\draw[color=qqqqff] (-4.74,-2.96) node {$D$};
		\draw [fill=white] (-1.5358683620315436,0.10984381782665315) circle (1.5pt);
		\draw[color=qqqqff] (-2.16,-0.1) node {$O$};
		\draw [fill=white] (-2.005126638597645,1.8904539905500792) circle (1.5pt);
		\draw [fill=white] (2.0712687778972603,1.544727100008277) circle (1.5pt);
		\draw [fill=white] (-1.5358683620315439,-2.6) circle (1.5pt);
		\draw [fill=white] (-4.116092792972028,0.5257993935953962) circle (1pt);
	\end{tikzpicture}
	\begin{flushright}
		\textit{George Apostolopoulos, Greece}
	\end{flushright}
\end{multicols}
\newpage
\centerline{{\large{\textbf{\color{thachthuctoanhoc}\color{thachthuctoanhoc}\color{thachthuctoanhoc}GIẢI BÀI KỲ TRƯỚC}}}}
\vspace*{-5pt}
\begin{multicols}{2}
	\setlength{\abovedisplayskip}{5pt}
	\setlength{\belowdisplayskip}{5pt}
	{\color{thachthuctoanhoc}{\usefont{T5}{qag}{b}{n} P761.}}
	(Mức $B$) Mỗi bạn An, Bình, Huệ, Nga đều có một khoản tiền tiết kiệm. Biết rằng, tiền tiết kiệm của tất cả các nhóm hai bạn, có thể lập được từ bốn bạn đó, là $1,9$ triệu đồng, $2,07$ triệu đồng, $2,11$ triệu đồng, $2,33$ triệu đồng, $2,5$ triệu đồng, và $x$ triệu đồng. Hãy tìm $x$.
	\vskip 0.05cm
	(Tiền tiết kiệm của một nhóm hai bạn là tổng tiền tiết kiệm của hai bạn đó.)
	\vskip 0.05cm
	\textbf{Lời giải} (\textit{dựa theo ý giải của bạn Lê Nguyễn Hoàng Nhật Đình, lớp $9$C, trường THCS Nguyễn Thái Bình, Tp. Cà Mau, tỉnh Cà Mau})\textbf{.}
	\vskip 0.05cm
	Gọi $a, b, c, d$ (triệu đồng), tương ứng, là số tiền tiết kiệm của các bạn An, Bình, Huệ, Nga.
	\vskip 0.05cm
	Theo giả thiết của bài ra, ta có:
	\begin{align*}
		\{a + b; a + c; a + d; b + c; b + d; c + d\} = \{1,9; 2,07; 2,11; 2,33; 2,5; x\}. \tag{$1$}
	\end{align*}
	Dễ thấy, có thể phân chia sáu số $a + b$, $a + c$, $a + d$, $b + c$, $b + d$, $c + d$ thành ba nhóm, mỗi nhóm hai số, sao cho các tổng hai số cùng nhóm bằng nhau (và cùng bằng $a + b + c + d$).
	\vskip 0.05cm
	Vì thế, theo ($1$), có thể phân chia sáu số $1,9$; $2,07$; $2,11$; $2,33$; $2,5$; $x$ thành ba nhóm, mỗi nhóm hai số, sao cho các tổng hai số cùng nhóm bằng nhau. \hfill ($2$)
	\vskip 0.05cm
	Suy ra, trong năm số $1,9$; $2,07$; $2,11$; $2,33$; $2,5$ phải có bốn số có tính chất: Có thể phân chia bốn số đó thành hai nhóm, mỗi nhóm hai số, sao cho các tổng hai số cùng nhóm bằng nhau. \hfill ($3$)
	\vskip 0.05cm
	Từ năm số nêu trên, có thể lập được năm nhóm bốn số. Bằng cách kiểm tra trực tiếp từng nhóm, trong năm nhóm đó, ta thấy chỉ có duy nhất nhóm bốn số $1,9$; $2,07$; $2,33$; $2,5$ có tính chất nêu ở ($3$); cụ thể, ta có:
	\begin{align*}
		1,9 + 2,5 = 2,07 + 2,33 = 4,4.
	\end{align*}
	Vì nhóm bốn số nói trên là nhóm duy nhất có tính chất nêu ở ($3$), nên từ đây và ($2$) suy ra, phải có
	\begin{align*}
		2,11 + x = 4,4;
	\end{align*}
	Vì thế, $x = 4,4 - 2,11 = 2,29$.
	\vskip 0.05cm
	Ta có điều cần tìm theo yêu cầu đề bài.
	\vskip 0.05cm
	\textbf{Bình luận và Nhận xét}
	\vskip 0.05cm	
	Trong số các lời giải Tạp chí đã nhận được từ bạn đọc, chỉ có hai lời giải đúng. Các lời giải còn lại là lời giải sai, do người giải bài đã ngộ nhận rằng, với các ký hiệu ở Lời giải trên và với giả sử $a + b = x$, ta có
	\begin{align*}
		(1) \Leftrightarrow \begin{cases}
			c\,\, + \,\,d\,\, \in \,\,\left\{ {1,9;\,\,2,07;\,\,2,11;\,\,2,33;\,\,2,5} \right\}\\
			x\,\, + \,\,c\,\, + \,\,d\,\, = \,\,\frac{{1,9\,\, + \,\,2,07\,\, + \,\,2,11\,\, + \,\,2,33\,\, + \,\,2,5\,\, + \,\,x}}{3}.
		\end{cases}
	\end{align*}
	\begin{flushright}
		\textbf{Nguyễn Khắc Minh}
	\end{flushright}
	{\color{thachthuctoanhoc}{\usefont{T5}{qag}{b}{n} P762.}}
	(Mức $B$) Cho tam giác $ABC$ với trọng tâm $G$. Dựng ra phía ngoài tam giác đó các tam giác $ABM$ và $ACN$, sao cho $\angle BAM = 45^\circ$, $AB = 3\sqrt{2}AM$, $\angle NAC = 90^\circ$, và $AC = 3AN$. Chứng minh rằng, $\angle GMN = 90^\circ$.
	\vskip 0.05cm  
	\textbf{Lời giải} (\textit{phỏng theo đa số lời giải Tạp chí đã nhận được từ bạn đọc})\textbf{.}
	\vskip 0.05cm
	
	%%%%
	Dựng ra phía ngoài tam giác $ABC$, tam giác $ABE$ vuông cân tại $E$, và tam giác $ACF$ vuông cân tại $A$.
	\vskip 0.05cm
	Gọi $I$ là trung điểm của $BC$; dựng hình bình hành $ABDC$.
	\vskip 0.05cm
	Từ việc dựng nêu trên và các giả thiết của bài ra, ta có:
	\vskip 0.05cm
	($i$) $M$ thuộc tia $AE$, và $\sqrt{2}AE = AB = 3\sqrt{2}AM$;
	\vskip 0.05cm 
	($ii$) $N$ thuộc tia $AF$, và $AF = AC = 3AN$;
	\vskip 0.05cm
	($iii$) $G$ thuộc tia $AD$, và $AG\,\, = \,\,\frac{2}{3}AI\,\, = \,\,\frac{2}{3} \cdot \left( {\frac{1}{2}AD} \right)\,\, = \,\,\frac{1}{3}AD$;
	\vskip 0.05cm
	($iv$) $BD = AC = AF$;
	\vskip 0.05cm
	($v$)  
	\begin{align*}
		\angle EBD\,\, = \,\,{45^{\rm{o}}}\, + \,\,\angle ABD\,\, = \,\,{45^{\rm{o}}}\, + \,\,\left( {{{180}^{\rm{o}}}\, - \,\,\angle CAB} \right)\,\, = \,\,{225^{\rm{o}}}\, - \,\,\angle CAB\\
			= \,\,\left( {{{360}^{\rm{o}}}\, - \,\,\left( {{{45}^{\rm{o}}}\, + \,\,{{90}^{\rm{o}}}} \right)} \right)\,\, - \,\,\angle CAB\\
			= \,\,{360^{\rm{o}}}\, - \,\,\left( {{{45}^{\rm{o}}}\, + \,\,\angle CAB\,\, + \,\,{{90}^{\rm{o}}}} \right)\,\, = \,\,\angle EAF.
	\end{align*}
	Từ ($i$) và ($ii$) suy ra, $M$ thuộc tia $AE$, $N$ thuộc tia $AF$, và $\dfrac{AM}{AE} = \frac{AN}{AF}$. Vì thế, theo định lý Thales,
	\begin{align*}
		MN \parallel EF.       \tag{$1$}                 
	\end{align*}                          
	Từ ($i$) và ($iii$) suy ra, $M$ thuộc tia $AE$, $G$ thuộc tia $AD$, và $\dfrac{AM}{AE} = \dfrac{AG}{AD}$  Vì thế, theo định lý Thales,
	\begin{align*}
		MG \parallel ED.  \tag{$2$}
	\end{align*}                  
	Từ ($1$) và ($2$) suy ra, hai góc $GMN$ và $DEF$ hoặc bằng nhau, hoặc bù nhau.                                           \hfill  ($3$)
	\vskip 0.05cm
	Tiếp theo, do tam giác $ABE$ cân tại $E$ nên $BE = AE$. Từ đây và ($iv$), ($v$), suy ra 
	\begin{align*}
		\Delta EBD = \Delta EAF.
	\end{align*}
	Do đó, $\angle BED = \angle AEF$. Vì thế
	\begin{align*}
		\angle DEF\,\, = \,\,\angle BEF\,\, - \,\,\angle BED\,\, = \,\,\left( {\angle BEA\,\, + \,\,\angle AEF} \right)\,\, - \,\,\angle BED\,\, = \,\,\angle BEA\,\, = \,\,{90^{\circ}}. \tag{$4$}
	\end{align*}
	Từ ($3$) và ($4$), suy ra  $\angle GMN = 90^\circ$.
	\vskip 0.05cm
	Ta có điều phải chứng minh theo yêu cầu đề bài.
	\vskip 0.05cm
	\textbf{Bình luận và Nhận xét}
	\vskip 0.05cm
	Tất cả các lời giải Tạp chí đã nhận được từ bạn đọc đều là lời giải đúng.
	\begin{flushright}
		\textbf{Hạ Vũ Anh}
	\end{flushright}
	{\color{thachthuctoanhoc}{\usefont{T5}{qag}{b}{n} P763.}}
	(Mức $B$) Cho một đa giác đều có $2024$ đỉnh; tại mỗi đỉnh, người ta viết một số nguyên dương không vượt quá $1011$. Chứng minh rằng, tồn tại bốn đỉnh $A$, $B$, $C$, $D$ của đa giác đó, sao cho $ABCD$ là một hình chữ nhật, và $a + b = c + d$; trong đó, $a$, $b$, $c$, $d$, tương ứng, là số được viết tại các đỉnh $A$, $B$, $C$, $D$.
	\vskip 0.05cm
	\textbf{Lời giải} (\textit{dựa theo ý giải của bạn Lê Nguyễn Hoàng Nhật Đình, lớp $9$C, trường THCS Nguyễn Thái Bình, Tp. Cà Mau, tỉnh Cà Mau})\textbf{.}
	\vskip 0.05cm
	Ký hiệu ($H$) là đa giác đều đã cho trong đề bài.
	\vskip 0.05cm
	Vì ($H$) là đa giác đều, nên nó có đường tròn ngoại tiếp; ký hiệu đường tròn này là ($O$).
	\vskip 0.05cm
	Vì đa giác đều ($H$) có $2024$ đỉnh nên ($O$) có $1012 (= 2024 : 2)$ đường kính đôi một khác nhau, mà mỗi đường kính đều có cả hai đầu mút là đỉnh của ($H$). Vì vậy, ($O$) có $1012$ đường kính, mà ở mỗi đầu mút của mỗi đường đều có ghi một số nguyên dương không vượt quá $1011$.
	\vskip 0.05cm
	Xét $1012$ đường kính vừa nêu trên.
	\vskip 0.05cm
	Gán cho mỗi đường kính một số, bằng trị tuyệt đối của hiệu của hai số được ghi ở hai đầu mút của đường kính ấy.
	\vskip 0.05cm
	Khi đó, do số được ghi ở mỗi đầu mút của mỗi đường kính là một số nguyên dương không vượt quá $1011$, nên mỗi đường kính được gán một số tự nhiên không vượt quá $1010 (= 1011 - 1)$.
	\vskip 0.05cm
	Từ đó, do trong phạm vi từ $0$ đến $1010$ chỉ có $1011$ số tự nhiên đôi một khác nhau, và $1012 > 1011$, nên trong các số đã gán cho $1012$ đường kính phải có ít nhất hai số bằng nhau. Nói cách khác, trong số $1012$ đường kính phải có hai đường kính được gán cùng một số; gọi hai đường kính này là ($I$), ($II$).
	\vskip 0.05cm
	Ký hiệu $a$, $c$, với $a \ge c$, là hai số được ghi ở hai đầu mút của ($I$); và gọi $A$ là đầu mút được ghi số $a$, $C$ là đầu mút được ghi số $c$.
	\vskip 0.05cm
	Ký hiệu $b$, $d$, với $d \ge b$, là hai số được ghi ở hai đầu mút của ($II$); và gọi $B$ là đầu mút được ghi số $b$, $D$ là đầu mút được ghi số $d$.
	\vskip 0.05cm
	Khi đó:
	\vskip 0.05cm
	-- Vì tất cả các đầu mút của ($I$) và ($II$) đều là đỉnh của ($H$), và vì ($I$), ($II$) là hai đường kính khác nhau của cùng một đường tròn, nên $A$, $B$, $C$, $D$ là bốn đỉnh của ($H$) và tứ giác $ABCD$ là một hình chữ nhật;
	\vskip 0.05cm
	-- $|a - c|$ và $|b - d|$, tương ứng, là số được gán cho đường kính $AC$ và đường kính $BD$. Do đó
	\begin{align*}
		a - c = |a - c| = |b - d| = d - b.
	\end{align*}
	Suy ra, $a + b = c + d$.
	\vskip 0.05cm
	Vì vậy, ta có điều phải chứng minh theo yêu cầu đề bài.
	\vskip 0.05cm
	\textbf{Bình luận và Nhận xét}
	\vskip 0.05cm
	Tất cả các bạn đọc đã gửi lời giải tới Tạp chí đều có ý tưởng giải đúng. Tuy nhiên, ngoại trừ bạn \textit{Lê Nguyễn Hoàng Nhật Đình}, tất cả các bạn còn lại đều diễn đạt rất chưa ổn quá trình triển khai thực hiện ý tưởng giải của mình. Chẳng hạn, có bạn đã viết: Ở hai đầu của đường kính có ghi hai số $x$, $y$, nên đường kính có độ dài là $|x - y|$; ...
	\begin{flushright}
		\textbf{Nguyễn Khắc Minh}
	\end{flushright}
	{\color{thachthuctoanhoc}{\usefont{T5}{qag}{b}{n} P763.}}
	(Mức $B$) Tìm tất cả các bộ ba số thực $(x, y, z)$ thỏa mãn: $x,\,\,y,\,\,z\,\, \in \,\,\left( {0;\,\,\frac{1}{2}} \right)$  và
	\begin{cases}
			\left( {3{x^2}\, + \,\,{y^2}} \right) \sqrt{1\,\, - \,\,4{z^2}} \,\, \ge \,\,z\\
			\left( {3{y^2}\, + \,\,{z^2}} \right)\sqrt {1\,\, - \,\,4{x^2}} \,\, \ge \,\,x\\
			\left( {cases3{z^2}\, + \,\,{x^2}} \right)\sqrt {1\,\, - \,\,4{y^2}} \,\, \ge \,\,y.
	\end{cases}
	Lời giải (dựa theo Đáp án của BBT Tạp chí).
	• Giả sử (x, y, z) là bộ ba số thực thỏa mãn các yêu cầu của đề bài.
	Do   nên từ bất đẳng thức thứ nhất của hệ đã cho trong đề bài, ta có:
	
	Suy ra
	(1)
	dấu “=” xảy ra khi và chỉ khi
	
	Bằng cách hoàn toàn tương tự, từ bất đẳng thức thứ hai, bất đẳng thức thứ ba của hệ đã cho trong đề bài, ta được:
	(2)
	dấu “=” xảy ra khi và chỉ khi
	
	(3)
	dấu “=” xảy ra khi và chỉ khi
	
	Cộng ba bất đẳng thức (1), (2), (3), vế theo vế, ta được:
	(4)
	dấu “=” xảy ra khi và chỉ khi dấu “=” xảy ra đồng thời ở (1), (2) và (3).
	Từ đó, do ở (4) xảy ra dấu “=” nên ở (1), (2), (3) phải đồng thời xảy ra dấu “=”.
	Từ các điều kiện cần và đủ để xảy ra dấu “=” ở mỗi bất đẳng thức (1), (2), (3), đã nêu ở trên, dễ thấy, dấu “=” đồng thời xảy ra ở ba bất đẳng thức đó khi và chỉ khi
	
	Như vậy, nếu (x, y, z) là bộ ba số thực thỏa mãn các yêu cầu của đề bài thì
	(5)
	• Ngược lại, bằng cách kiểm tra trực tiếp, dễ thấy, bộ ba số thực (x, y, z), được nêu ở (5), thỏa mãn các yêu cầu của đề bài.
	Vậy, có duy nhất bộ ba số thực cần tìm theo yêu cầu đề bài: bộ (x, y, z) được nêu ở (5).
	Bình luận và Nhận xét
	Tất cả các lời giải Tạp chí đã nhận được từ bạn đọc, rất tiếc, đều là lời giải sai, do người giải bài đã ngộ nhận rằng, không mất tính tổng quát, có thể giả sử x  y  z.
	Nguyễn Khắc Minh
	P765. (Mức B) Cho số nguyên dương n. Gọi A là tích tất cả các số nguyên dương không vượt quá 2n + 1, B là tích tất cả các số nguyên dương lẻ không vượt quá 2n + 1, và C là tích tất cả các số nguyên dương chẵn không vượt quá 2n. Chứng minh rằng,   không là số chính phương.
	Lời giải (dựa theo ý giải của bạn Đỗ Duy Quang, lớp 12T1, trường THPT chuyên Nguyễn Quang Diêu, tỉnh Đồng Tháp).
	Đặt  
	Do   (vì mỗi số đều là tích của các số nguyên dương), nên S là một số nguyên dương.
	Từ giả thiết của bài ra, hiển nhiên có A = BC.
	Do đó
	
	Theo giả thiết của bài ra, ta có:
	
	Từ đó, do   nên B - C \ge 1. Vì vậy, từ (1) ta được:
	(3)
	(2) và (3) cho thấy, số nguyên dương S nằm giữa hai số chính phương liên tiếp. Vì thế, S không là số chính phương.
	Ta có điều phải chứng minh theo yêu cầu đề bài.
	Bình luận và Nhận xét
	1. Lời giải trên cho thấy, bài đã ra là một trường hợp đặc biệt của kết quả sau:
	“Nếu x, y là các số nguyên dương, thỏa mãn x > y, thì
	
	không là số chính phương.”
	2. Lời giải của bạn Đỗ Duy Quang là lời giải duy nhất Tạp chí đã nhận được từ bạn đọc.
	Lưu Thị Thanh Hà
	P766. (Mức B) Cho a, b, c là các số dương. Chứng minh rằng
	
	Lời giải (của người chấm bài).
	Kí hiệu A là vế trái của bất đẳng thức cần chứng minh theo yêu cầu đề bài.
	Đặt
	
	Do a, b, c  0 (theo giả thiết) nên
	
	(theo bất đẳng thức Cauchy - Schwarz).
	Do a, b, c > 0 (giả thiết), nên
	
	Ta có điều phải chứng minh theo yêu cầu đề bài.
	Bình luận và Nhận xét
	1. Từ lời giải trên, dễ thấy, dấu “=” ở bất đẳng thức của đề bài xảy ra khi và chỉ khi a = b = c.
	2. Có nhiều hướng tiếp cận để chứng minh bất đẳng thức của đề bài. Các hướng được sử dụng nhiều trong các lời giải Tạp chí đã nhận được từ bạn đọc, là:
	- Đưa việc chứng minh bất đẳng thức của đề bài về việc chứng minh bất đẳng thức
	
	bằng cách đặt      
	- Sử dụng các bất đẳng thức
	
	
	3. Trong số các lời giải Tạp chí đã nhận được từ bạn đọc, rất tiếc, có một lời giải sai, do người giải bài đã mắc lỗi sai kiến thức cơ bản về bất đẳng thức.
	4. Kết quả dưới đây là một trong các khái quát có thể từ bài đã ra:
	Kết quả khái quát. Với mọi số thực dương a, b, c, và với mọi số thực không âm k, ta có:
	
	Bạn đọc có thể chứng minh kết quả trên, bằng cách sử dụng kết quả sau của Vasile Cirtoaje:
	“Với mọi số thực a, b, c, ta đều có:
	”
	Võ Quốc Bá Cẩn
	P767. (Mức A) Cho x, y, z là các số thực không âm, thỏa mãn   Chứng minh rằng
	
	Lời giải (của người chấm bài).
	Trong phần trình bày dưới đây, cụm từ “trung bình cộng - trung bình nhân” được viết tắt là “tbc - tbn”.
	Không mất tính tổng quát, giả sử x \ge y \ge z.
	Khi đó, (y - z)(y - x)  0; do đó
	
	Suy ra
	
	Vì thế
	
	Ta có điều phải chứng minh theo yêu cầu đề bài.
	Bình luận và Nhận xét
	1. Trong Lời giải trên, ta đã sử dụng bất đẳng thức Schur bậc ba ở dạng:
	
	2. Dễ thấy, dấu “=” ở bất đẳng thức của đề bài xảy ra khi và chỉ khi hai trong ba số x, y, z bằng   và số còn lại bằng 0.
	3. Ngoài cách đã trình bày ở Lời giải trên, còn có thể chứng minh bất đẳng thức
	
	bằng cách sử dụng đồng nhất thức
	
	và bất đẳng thức trung bình cộng - trung bình nhân.
	4. Bài đã ra là một bài toán khó. Hầu hết các lời giải Tạp chí đã nhận được từ bạn đọc đều khá cồng kềnh, có nhiều tính toán phức tạp.
	5. Trong số các lời giải Tạp chí đã nhận được từ bạn đọc, rất tiếc, có một lời giải sai, do người giải bài đã mắc lỗi sai kiến thức cơ bản về bất đẳng thức.
	Võ Quốc Bá Cẩn
	P768. (Mức A) Cho a là một số nguyên dương lẻ và không là số chính phương. Chứng minh rằng
	
	với mọi số nguyên dương m, n.
	Lời giải (dựa theo Đáp án của BBT Tạp chí).
	Giả sử, ngược lại, tồn tại các số nguyên dương m, n, sao cho
	(1)
	Đặt  
	Do a là số nguyên dương không chính phương (giả thiết), nên   và   là các số vô tỉ. Vì thế
	
	và do (1) nên
	
	Suy ra
	(2)
	(3)
	Cộng các bất đẳng thức kép (2) và (3), vế theo vế, ta được:
	
	suy ra
	
	là điều vô lí, do với a là số nguyên dương lẻ không chính phương,   là một số nguyên dương.
	Điều vô lý nhận được ở trên cho ta điều phải chứng minh theo yêu cầu đề bài.
	Bình luận và Nhận xét
	1. Bài đã ra là một trường hợp đặc biệt của kết quả kinh điển sau:
	“Nếu ,  là các số vô tỉ dương, có tổng nghịch đảo bằng 1 (tức  ), thì với mọi    ”
	Cụ thể, bạn sẽ thu được kết luận của bài đã ra, từ kết quả kinh điển nêu trên, khi chọn   và   với a là một số nguyên dương lẻ không chính phương.
	2. Trong số các lời giải Tạp chí đã nhận được từ bạn đọc, rất tiếc, có một lời giải sai, do người giải bài đã ngộ nhận rằng, với x, y là các số vô tỉ dương, nếu [x] = [y] thì x = y.
	Lưu Thị Thanh Hà
	P769. (Mức A) Cho tam giác không cân ABC nội tiếp đường tròn (O). Trên tiếp tuyến tại A của (O), lấy điểm K tùy ý, khác A. Một đường thẳng , đi qua K, cắt các đường thẳng BC, CA, AB, tương ứng, tại D, E, F. Chứng minh rằng, tâm đẳng phương của đường tròn đường kính EF, đường tròn đường kính DK và đường tròn (O) nằm trên đường thẳng .
	Lời giải (dựa theo lời giải của bạn Vương Khánh Toàn, lớp 11A1 Toán, trường THPT chuyên Tự nhiên, ĐH Khoa học Tự nhiên - ĐHQG Hà Nội).
	Gọi I, J, tương ứng, là trung điểm của EF, DK; ta có, I, J, tương ứng, là tâm đường tròn đường kính EF, đường tròn đường kính DK.
	Kí hiệu (I), (J), tương ứng, là đường tròn đường kính EF, đường tròn đường kính DK.
	Theo bài ra, ta phải chứng minh tâm đẳng phương của ba đường tròn (I), (J), (O) thuộc đường thẳng .
	Vì bài ra đề cập đường tròn đường kính EF, nên ta hiểu E, F phải là hai điểm phân biệt. Do đó, đường thẳng  phải không trùng với tiếp tuyến AK của (O). Vì vậy, chỉ có thể xảy ra hai trường hợp sau đối với vị trí của :
	-  đi qua tâm O của (O);
	-  không đi qua O và không trùng với AK.
	• Xét trường hợp 1:  đi qua tâm O của (O) (xem Hình 1).
	
	Hình 1
	Trong trường hợp này, hiển nhiên, ba điểm I, J, O thẳng hàng. Vì thế, ba đường tròn (I), (J), (O) không có tâm đẳng phương.
	• Xét trường hợp 2:  không đi qua O và không trùng với AK.
	Gọi M là giao điểm thứ hai, khác A, của (O) và đường tròn (AEF).
	
	Hình 2
	Ta có, M là điểm Miquel của tứ giác toàn phần BCEFAD. Do đó, M thuộc đường tròn (CDE). Vì vậy
	
	Suy ra, bốn điểm A, K, D, M cùng thuộc một đường tròn.
	Xảy ra hai trường hợp sau:
	• Trường hợp 2.1: AM \parallel  (xem Hình 3).
	
	Hình 3
	Trong trường hợp này, do AMDK và AMEF là các tứ giác nội tiếp, nên chúng là các hình thang cân. Do đó, I, J cùng thuộc đường trung trực của AM; suy ra, I  J và OI  . Vì thế, ba đường tròn (I), (J), (O) không có tâm đẳng phương.
	• Trường hợp 2.2: AM cắt  (xem Hình 2).
	Gọi T là giao điểm của AM và . Xét phương tích của T đối với các đường tròn, ta có:
	
	Do đó, T là tâm đẳng phương của ba đường tròn (I), (J), (O). Từ đây, do T thuộc  nên ta có điều phải chứng minh theo yêu cầu đề bài.
	Bình luận và Nhận xét
	1. Lời giải trên cho thấy, để đảm bảo tính chuẩn xác của một bài toán, kết luận của bài ra nên được phát biểu là: Chứng minh rằng, nếu đường tròn đường kính EF, đường tròn đường kính DK và đường tròn (O) có tâm đẳng phương thì tâm ấy nằm trên đường thẳng .
	2. Lời giải của bạn Vương Khánh Toàn là lời giải duy nhất, trong số các lời giải Tạp chí đã nhận được từ bạn đọc, đề cập đầy đủ các trường hợp có thể xảy ra đối với vị trí của đường thẳng .
	Hạ Vũ Anh
	P770. (Mức A) Cho phép thực hiện việc đổi chỗ các số hạng trong một hoán vị, theo qui tắc: Mỗi lần, lấy ra khỏi hoán vị tám số hạng tùy ý của nó, rồi lại xếp tám số hạng đó vào tám vị trí mà chúng đã nằm, nhưng theo thứ tự ngược lại.
	Hỏi, nhờ việc thực hiện liên tiếp một số hữu hạn lần phép đổi chỗ nói trên đối với hoán vị (1, 2, …, 2023) của 2023 số nguyên dương đầu tiên, ta có thể nhận được hoán vị (2023, 2022, …, 1) hay không?
	Lời giải (dựa theo ý giải của bạn Trần Minh Hoàng, lớp 11T1, trường THPT chuyên Hà Tĩnh, tỉnh Hà Tĩnh).
	Gọi   là phép đổi chỗ đã cho trong đề bài, và gọi   là phép đổi chỗ hai số hạng tùy ý trong một hoán vị.
	Giả sử n là một số nguyên dương lớn hơn 1, và   là một hoán vị của n số thực đôi một khác nhau.
	Ta gọi cặp số hạng   là một nghịch thế của hoán vị a, nếu i  j và  
	Với i, j là hai số khác nhau thuộc tập hợp {1; 2; …; n},   kí hiệu việc hoán đổi vị trí của   cho nhau, trong hoán vị a.
	Xét hoán vị   bất kì của 2023 số nguyên dương đầu tiên.
	Ta có các Nhận xét sau:
	Nhận xét 1. 
	Bình luận và Nhận xét
	1. Từ Lời giải trên dễ thấy, kết quả của bài ra không thay đổi, nếu trong giả thiết thay 8 bởi   với k là số nguyên dương tùy ý lớn hơn 1 và không vượt quá  
	2. Lời giải của bạn Trần Minh Hoàng là lời giải duy nhất Tạp chí nhận được từ bạn đọc.
	Nguyễn Khắc Minh
\end{multicols}


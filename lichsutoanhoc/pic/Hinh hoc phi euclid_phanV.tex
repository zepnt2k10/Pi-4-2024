
\documentclass{article}
\usepackage{graphicx} % Required for inserting images
\usepackage[utf8]{vietnam} 
\title{Hình phi Euclid\\
Phần IV: Hình học trên đĩa Poincar\'e}
\date{May 2023}

\begin{document}

\maketitle

\begin{figure}[H]
		\vspace*{-5pt}
		\centering
		\captionsetup{labelformat= empty, justification=centering}
		\includegraphics[width= 0.9\linewidth]{Klein sang Poincare.png}
%		\caption{\small\textit{\color{}}}
		\vspace*{-10pt}
	\end{figure}
	Dựng một mặt cầu tiếp xúc với mặt phẳng chứa đường tròn $\gamma$ của mô hình Beltrami--Klein tại tâm đường tròn. Để chuyển đổi từ mô hình Klein sang mô hình Poincare, ta chiếu vuông góc dây mở trong mô hình Klein lên mặt cầu. Từ điểm $N$ trên đỉnh mặt cầu, với mỗi điểm $M$ trên hình chiếu vuông góc của dây mở lên mặt cầu, ta chiếu $M$ trở lại mặt phẳng chứa $\gamma$ thành điểm M' là giao điểm của đường $NM$ với mặt phẳng ấy. Khi này một dây mở của mô hình Klein sẽ trở thành một đường $P$ trong đường tròn đồng tâm và bán kính gấp đôi $\gamma$. Ta vị tự tâm $O$ tỷ số $\dfrac{1}{2}$ là đưa đường tròn mới về $\gamma$.
	\vskip 0.1cm
	Tiếp theo là định nghĩa độ dài đoạn trong mô hình Poincaré:
	\vskip 0.1cm
	Định nghĩa: Với hai điểm $A$, $B$ nằm bên trong đường tròn gamma và $M$, $N$ là hai ``giao điểm'' của đường $P$ đi qua $A$, $B$ với gamma. Khi ấy độ dài trong mô hình Poincaré của đoạn $AB$ là:
	\begin{align*}
		 d(AB) = |\ln{(AB, MN)}|.
	\end{align*} 
	Định nghĩa của tỷ số kép cho bộ $4$ điểm cùng nằm trên một đường thẳng/đường tròn đã được nêu ở phần trên. Không khó để chỉ ra là tỷ số kép này là số thực dương, do $A$, $B$ nằm bên trong đường tròn $\gamma$. Đừng để ký hiệu logarit thêm vào làm cho hoang mang, nó ở đó chỉ để biến tính nhân thành tính cộng, điều này sẽ rõ trong chứng minh. Giá trị tuyệt đối được thêm vào để bất kể thứ tự của $P$ và $Q$, của $A$ và $B$, độ dài ko đổi dấu sang âm.
	\vskip 0.1cm
	Ta định nghĩa đoạn $AB$ bằng đoạn $CD$ trong hình hyperbolic nếu như $d(AB) = d(CD)$. Không khó có được C$2$ ngay. 
	Cố định $1$ điểm $A$ trên đường $P$ có hai đầu là $M$ và $N$, để điểm $B$ chạy liên tục từ $A$ tới $M$ sao cho thứ tự các điểm là $N$, $A$, $B$, $M$ như hình. tỷ số kép ($AB, MN$) sẽ tăng liên tục từ $1$ đến dương vô cùng, do $\frac{AM}{AN}$ cố định, $BM$ tiến tới $0$ và $BN$ tiến tới $MN$. Cố định $B$ và cho $A$ chạy liên tục từ $B$ tới $N$ cũng tương tự. Do chạy liên tục nên tỷ số kép trên có thể nhận mọi giá trị lớn hơn hoặc bằng $1$, ta chứng minh được C$1$. 
	\begin{figure}[H]
		\vspace*{-5pt}
		\centering
		\captionsetup{labelformat= empty, justification=centering}
		\includegraphics[width= 0.9\linewidth]{Tiên đề NTĐQ 1.pdf}
%		\caption{\small\textit{\color{}}}
		\vspace*{-10pt}
	\end{figure}
	C$3$ sẽ đúng một khi chứng minh được là với điểm $C$ nằm giữa $A$ và $B$ (trong mô hình Poincaré) thì $d(AC) + d(CB) = d(AB)$. Lấy hai đầu $M$, $N$ của đường $P$ qua $A$ và $B$ để thứ tự các điểm là $N - A - C - B - M$.
	\vskip 0.1cm
	Khi ấy các tỷ số kép sắp được đề cập đều lớn hơn $1$ và ta có thể bỏ dấu giá trị tuyệt đối. Lúc này:
	\begin{align*}
		d(AC) \!+\! d(CB) &= \ln{(AC, MN)} \!+\! \ln{(CB, MN)} \\
		&= \ln{\left(\frac{MA}{MC} \!:\! \frac{NA}{NC} . \frac{MC}{MB} \!:\! \frac{NC}{NB} \right)} \\
		&= \ln{ \left(\frac{MA}{MB} : \frac{NA}{NB}\right)} \\
		&= \ln{(AB, MN)} = d(AB)
	\end{align*}
	chính là điều ta cần.
	\vskip 0.1cm
	Về phần các tiên đề về sự bằng nhau của góc, ta cần nói qua về phép nghịch đảo và các tính chất liên quan tới hai đường tròn trực giao.
	\vskip 0.1cml
	$\pmb{8.}$ \textbf{\color{lichsutoanhoc}Mở rộng của định nghĩa góc. Sơ lược về phép nghịch đảo}
	\vskip 0.1cm
	Trước tiên ta mở rộng khái niệm góc trong hình Euclid. Có các trường hợp sau đây:
	Đường thẳng $d$ với đường tròn $\Omega$ : nếu chúng không có giao điểm thì sẽ không tạo góc với nhau. Nếu có giao điểm $X$, thì góc giữa $d$ và $\Omega$  được định nghĩa là góc giữa $d$ và tiếp tuyến của $\Omega$  tại $X$.
	Giữa hai đường tròn $(O1)$ và $(O2)$, hay giữa hai đường cong có tiếp tuyến tại giao điểm: nếu chúng có giao điểm là $Y$ thì góc giữa chúng là góc giữa hai tiếp tuyến của chúng tại $Y$. Đặc biệt nếu góc ấy là góc vuông thì ta gọi hai đường tròn là trực giao. Định nghĩa này phù hợp với định nghĩa ta đưa ra lúc trước. 
	\vskip 0.1cm
	Phép nghịch đảo là một phép biến hình trong hình học Euclid, được ghi nhận lần đầu tiên trong công trình của Apollonius xứ Perga, thuộc Hy Lạp cổ đại, và được phát triển sâu hơn bởi Jakob Steiner và August Mobius, cùng nhiều người khác.
	\vskip 0.1cm
	Với một điểm $O$ và số thực $k$ khác $0$ bất kỳ, định nghĩa phép nghịch đảo tâm $O$ phương tích $k$ là phép biến mỗi điểm $Q$ khác $O$ trên mặt phẳng thành điểm $Q'$ sao cho $Q'$ nằm trên đường thẳng đi qua $O$, $Q$, và $\overline{OQ}.\overline{OQ'} = k$. Nói chung ta cũng ký hiệu $Q'$ là ảnh của điểm $Q$ qua phép nghịch đảo ta đang xét trong văn cảnh.
	Phép nghịch đảo qua đường tròn tâm $O$ bán kính $R$ được định nghĩa là phép nghịch đảo tâm $O$ phương tích $R^2$.
	\vskip 0.1cm
	Khi một điểm $P$ dịch chuyển gần lại về tâm nghịch đảo $O$ theo phương một đường thẳng nào đó, ảnh $P'$ sẽ dịch chuyển xa khỏi $O$ cũng theo phương đó. Vì thế một số người quy ước ảnh của điểm $O$ qua phép nghịch đảo là ``điểm vô cùng''. Tuy nhiên bởi phương đường thẳng này là chọn tùy ý nên ``điểm vô cùng'' này sẽ khác $O$ và nằm trên mọi đường thẳng đi qua $O$, vì thế sẽ khác so với các điểm vô cùng trong mặt phẳng xạ ảnh. Ở đây chúng ta sẽ không xét đến ảnh của điểm $O$ để đỡ phức tạp, một vài kết quả trong phần khảo sát ảnh sẽ hơi khác so với những gì độc giả có thể đã quen biết.
	\vskip 0.1cm
	Sau đây là một số tính chất có thể được dễ dàng suy ra từ định nghĩa và xét các tam giác đồng dạng theo trường hợp c.g.c:
	\vskip 0.1cm
	-- Phép nghịch đảo là một involution(tự nghịch/đối hợp), tức là tác động cùng một phép nghịch đảo hai lần thì trở về trạng thái ban đầu (hợp hai lần của một phép nghịch đảo là phép đồng nhất). Nói cách khác thì một phép nghịch đảo có hàm ngược là chính nó.
	Tổng quát hơn thì hợp của hai phép nghịch đảo cùng tâm là một phép vị tự tâm đó 
	\vskip 0.1cm
	-- Tập hợp tất cả các điểm bất động qua phép này là đường tròn nghịch đảo.
	\vskip 0.1cm
	-- Cách dựng ảnh của một điểm qua phép nghịch đảo đường tròn:
	\vskip 0.1cm
	$+)$ $P$ nằm trong đường tròn: vẽ dây $TU$ đi qua $P$ và vuông góc $OP$, $P'$ chính là giao điểm của hai tiếp tuyến tại $U$ và $T$ của $(O, R)$
	\vskip 0.1cm
	$+)$ $P$ nằm ngoài đường tròn: vẽ $2$ tiếp tuyến $PT$, $PU$ tới đường tròn. $P' = TU$ cắt $OP$
	\vskip 0.1cm
	-- Phép nghịch đảo biến $P$ thành $P'$, $Q$ thành $Q'$, và $O$, $P$, $Q$ không thẳng hàng, thì $\Delta  P'OQ' \sim  \Delta POQ$.
	\vskip 0.1cm
	-- $A'B' = |k|. \frac{AB}{OA.OB}$
	\vskip 0.1cm
	-- Ảnh của một đường tròn trực giao với đường tròn nghịch đảo là chính nó.
	\begin{figure}[H]
		\vspace*{-5pt}
		\centering
		\captionsetup{labelformat= empty, justification=centering}
		\includegraphics[width= 0.9\linewidth]{Dựng ảnh phép nghịch đảo.pdf}
%		\caption{\small\textit{\color{}}}
		\vspace*{-10pt}
	\end{figure}
	Tiếp theo ta sẽ tìm hiểu hình dạng ảnh của đường thẳng và đường tròn qua phép nghịch đảo đường tròn. Vẫn xét phép nghịch đảo qua $(O, R)$ như trên: 
	\vskip 0.1cm
	-- Một đường thẳng $l$ đi qua $O$ sẽ biến thành $l$
	\vskip 0.1cm
	-- Một đường thẳng $l$ không đi qua $O$ sẽ biến thành  đường tròn đường kính $OA'$, với $A$ là chân đường vuông góc từ $O$ xuống $l$.
	\vskip 0.1cm 
	-- Một đường tròn đi qua $O$ sẽ biến thành đường thẳng $l$ vuông góc với $OA'$ tại $A'$, trong đó $OA$ là đường kính đường tròn trên. Do tính chất tự nghịch của phép nghịch đảo, điều này sẽ có ngay sau khi chứng minh tính chất thứ $2$. 
	\vskip 0.1cm
	-- Một đường tròn $(O1)$ không đi qua $O$ sẽ biến thành một đường tròn $(O2)$ không đi qua $O$. $O$ cũng là một tâm vị tự của hai đường tròn. 
	\vskip 0.1cm
	Qua đây ta thấy phép nghịch đảo hầu như không bảo toàn hình dạng của đường thẳng và đường tròn, nhưng nó bảo toàn được hai đặc tính quan trọng khác là góc giữa hai hình và tỷ số kép của các bộ điểm. Thật vậy, phép nghịch đảo bảo toàn góc nhưng đảo ngược chiều của góc đó. Nó lại đồng thời bảo toàn các quan hệ nằm trên, đi qua, nằm giữa. Do đã có hệ thức về độ dài một đoạn qua phép nghịch đảo  $A'B' = |k|. \frac{AB}{OA.OB}$ , ta sẽ chứng minh được là tỷ số kép bảo toàn qua phép biến hình này. Giờ đây việc mở rộng định nghĩa của tỷ số kép và góc cho các đối tượng phổ quát hơn tỏ rõ sự hữu dụng, khi mà hai đường thẳng qua phép nghịch đảo có thể biến thành một đường thẳng, một đường tròn hoặc $2$ đường tròn, bộ $4$ điểm trên đường thẳng thì biến thành bộ $4$ điểm trên đường tròn. Điểm thú vị là trong mô hình Poincare, bảo toàn tỷ số kép dẫn đến bảo toàn độ dài Poincare.
	\vskip 0.1cm
	Sau đây là một tính chất quan trọng về đường tròn trực giao, mà chúng tôi sẽ thực sự chứng minh:
	\vskip 0.1cm
	\textbf{\color{lichsutoanhoc}Định lý $\pmb{1}$}: Cho đường tròn $(O, R)$ và điểm $P$ khác tâm và không nằm trên đường tròn. Đường tròn $(O')$ đi qua $P$ sẽ trực giao với $(O)$ $ \equiv$ $(O')$ đi qua $P'$ -- ảnh của $P$ khi nghịch đảo qua $(O)$.
	\begin{figure}[H]
		\vspace*{-5pt}
		\centering
		\captionsetup{labelformat= empty, justification=centering}
		\includegraphics[width= 0.9\linewidth]{Định lý 1.pdf}
%		\caption{\small\textit{\color{}}}
		\vspace*{-10pt}
	\end{figure}
	\textit{Chứng minh.} (Đảo) Giả sử $(O')$ đi qua $P'$, có ngay $O'$ thuộc trung trực $PP'$, từ việc $P$ nằm giữa $O$, $P'$, ta suy ra được $O'P<O'O$, nghĩa là điểm $O$ nằm ngoài $(O')$. Điều đó cho phép vẽ tiếp tuyến $OT$ tới $(O')$, từ đó chứng minh được tam giác $OPT$ đồng dạng tam giác $OTP'$. Từ tỷ lệ cạnh bằng nhau có là $OT^2 = OP.OP' = R^2$ (định nghĩa nghịch đảo), tức $T$ nằm trên $(O, R)$. Lại có $OTO'$ là góc vuông, ta có được sự trực giao.
	\vskip 0.1cm
	(Thuận) Gọi $T$, $U$ là hai giao điểm của $(O)$ và $(O')$ trực giao. Bởi hai tiếp tuyến tại $T$ và $U$ giao nhau tại $O$, $O$ nằm ngoài $(O')$. Vậy nên $OP$ cắt lại $(O')$ tại $Q$ khác $P$, từ tam giác đồng dạng lại có $OP.OQ = OT^2 = R^2$ cho thấy $Q$ là ảnh của $P$ qua nghịch đảo $(O, R)$.
	\vskip 0.1cm
	Hệ quả: Quỹ tích của tâm các đường tròn trực giao $(O')$ và đi qua $P$ chính là trung trực $PP'$.
	\vskip 0.1cm
	Đảo lại, với $l$ nằm ngoài $(O, R)$, hạ $OO'$ vuông góc $l$. $(O')$ trực giao $(O)$ cắt $OO'$ tại điểm gọi là $P$. Khi ấy quỹ tích ở trên chính là $l$.
	\vskip 0.1cm
	Định lý này cho phép ta một các dựng đường $P$ đi qua hai điểm $M$, $N$ không thẳng hàng với tâm của $\gamma$ như sau. Vẽ $M'$ là ảnh nghịch đảo $M$ qua $\gamma$, vẽ đường tròn qua ba điểm $M$, $M'$, $N$. Đường tròn ấy sẽ trực giao với $\gamma$ và ta có cung cần tìm. Do $M$, $N$ xác định duy nhất, $M'$ cũng thế, và đường tròn qua $3$ điểm này sẽ là duy nhất.
	\vskip 0.1cm
	$\pmb{9.}$ textbf{\color{lichsutoanhoc}Các tiên đề về sự bằng nhau}
	\vskip 0.1cm
	Đến đến lúc chứng minh các tiên đề còn lại về quan hệ bằng nhau trong mô hình Poincaré đúng là định lý trong hình Euclid.
	\vskip 0.1cm
	C$5$ là hiển nhiên do góc được đo như ở hình Euclid. C$4$ nói rằng với một góc cho trước và một tia $AB$ thì dựng được $C$ để $ \angle CAB$ bằng góc đó. Nếu $A$ trùng với tâm $ \gamma$ thì các đường $P$ đi qua $A$ chính là các dây mở, việc dựng sẽ như ở hình Euclid.
	\vskip 0.1cm
	Nếu không, bài toán quy về: cho trước một đường thẳng $l$ đi qua $A$ và không đi qua tâm $O$ của $\gamma$, khi ấy có đúng một đường tròn $(O')$ trực giao với $\gamma$ và tiếp xúc với $l$ tại $A$. 
	\vskip 0.1cm
	Định lý ta vừa chứng minh cho thấy $(O')$ sẽ phải đi qua $A'$, do đó $O'$ nằm trên trung trực $AA'$. Mặt khác $(O')$ tiếp xúc $l$ tại $A$, nên $O'$ nằm trên đường vuông góc với $l$ tại $A$. Tóm lại, $O'$ sẽ được xác định duy nhất là giao điểm của hai đường kể trên.
	\vskip 0.1cm
	Để sẵn sàng chứng minh C$6$ -- trường hợp bằng nhau c.g.c của các tam giác Poincaré, ta cần hai bước đệm nữa:
	\vskip 0.1cm
	Định lý $2$: Cho $(O)$ trực giao $(O')$. Khi ấy, phép nghịch đảo qua $(O)$ sẽ biến $(O')$ thành $(O')$ và biến phần bên trong $(O')$ thành chính nó. Đồng thời phép này bảo toàn các quan hệ nằm trên, đi qua, nằm giữa và bằng nhau theo mô hình Poincaré. Tương tự với một đường thẳng $d$ đi qua $O$, $d$ cũng gọi là trực giao với $(O)$, và phép biến hình là đối xứng qua $d$. Gọi chung hai phép này là phép $P$--đối xứng.
	\vskip 0.1cm
	Định lý về xây dựng phép $P$--đối xứng: Với mỗi hai điểm $A$, $B$ trong $\gamma$, tồn tại đường $P$ $\delta$ sao cho phép $P$--đối xứng với $\delta$ biến $A$ thành $B$. Giao điểm của đường $P$ nối $A$ và $B$ với $\delta$ là trung điểm theo nghĩa Poincaré của $AB$.
	\vskip 0.1cm
	Chứng minh Định lý xây dựng: Xét các trường hợp:
	\vskip 0.1cm
	-- Hai điểm cách đều $O$. Chọn delta là đường kính mở của gamma vuông góc với đường thẳng (theo nghĩa Euclid) $AB$
	\vskip 0.1cm
	-- Một trong hai điểm trùng với $O$, giả sử là $A$. Đường $P$ đi qua $A$, $B$ sẽ là đường kính mở qua $OB$. Chọn delta là ($CC'$), trong đó $C$ là trung điểm $OB$ theo nghĩa Poincaré. 
	\begin{figure}[H]
		\vspace*{-5pt}
		\centering
		\captionsetup{labelformat= empty, justification=centering}
		\includegraphics[width= 0.9\linewidth]{TH 3 định lý xây dựng.pdf}
%		\caption{\small\textit{\color{}}}
		\vspace*{-10pt}
	\end{figure}
	-- $A$ và $B$ không nằm trên cùng một đường kính của $\gamma$. Gọi $\gamma$' là đường $P$ đi qua $A$ và $B$, đây là một đường tròn trực giao với $\gamma$, gọi $X$, $Y$ là giao điểm của $\gamma$' và $\gamma$. Để ý rằng $A'$, $B'$ là giao điểm thứ hai của $OA$, $OB$ với $\gamma$'. Gọi $M$, $N$ là giao điểm của $OA$, $OB$ với $XY$. Từ $OX$, $OY$ là hai tiếp tuyến với gamma', ta chỉ ra được $(OM, AA') = -1 = (ON, BB')$, dẫn đến $AB$, $A'B'$ (có giao điểm do trường hợp $1$ đã xét) và $XY$ đồng quy tại điểm gọi là $T$. Chọn $\delta$ là $(T, \sqrt{TA.TB})$. 
	\begin{figure}[H]
		\vspace*{-5pt}
		\centering
		\captionsetup{labelformat= empty, justification=centering}
		\includegraphics[width= 0.9\linewidth]{TH 4 định lý xây dựng.pdf}
%		\caption{\small\textit{\color{}}}
		\vspace*{-10pt}
	\end{figure}
	-- $A$, $B$ cùng nằm trên một đường kính $d$ của gamma và không cách đều $O$. Lấy $X$, $Y$ trên cùng nửa đường tròn đường kính $d$ sao cho $AX$, $BY$ vuông góc $d$ (theo nghĩa Euclid). Tâm của $\delta$ là giao điểm của $XY$ với $AB$.
	\vskip 0.1cm
	\textbf{\color{lichsutoanhoc}Trường hợp bằng nhau cạnh--góc--cạnh}
	\vskip 0.1cm
	Ta phát biểu lại trường hợp bằng nhau này:
	Cho hai tam giác ABC và XYZ có: $ \angle A = \angle X, d(AB) = d(XY), d(AC) = d(XZ)$.
	Khi ấy hai tam giác này bằng nhau, nghĩa là $d(BC)  = d(YZ), \angle B = \angle Y, góc C = góc Z$
	\begin{figure}[H]
		\vspace*{-5pt}
		\centering
		\captionsetup{labelformat= empty, justification=centering}
		\includegraphics[width= 0.9\linewidth]{Trường hợp cgc.pdf}
%		\caption{\small\textit{\color{}}}
		\vspace*{-10pt}
	\end{figure}
	\textit{Chứng minh.} Đầu tiên ta đưa bài toán về trường hợp đặc biệt $A$ và $X$ trùng $O$.
	Theo chứng minh định lý xây dựng phép P-đối xứng, có một phép nghịch đảo biến $A$ thành $O$ nếu $A$ khác $O$. Phép ấy sẽ biến $\Delta ABC$ thành $\Delta OB'C'$, và do tính bảo toàn góc và độ dài cạnh kiểu Poincaré của phép này, hai tam giác này bằng nhau. Để ý quan hệ bằng nhau của tam giác sẽ có tính bắc cầu từ việc quan hệ bằng nhau của góc và cạnh là bắc cầu. Một cách tương tự ta đẩy $\Delta XYZ$ về $ \Delta OY'Z'$.
	\vskip 0.1cm
	Bổ đề sau đây sẽ chuyển sự bằng nhau của các đoạn Poincare thành đoạn Euclid:
	\vskip 0.1cm
	Bổ đề: $OB = r.(e^d -1)/(e^d+1)$, ở đây $r$ là bán kính của gamma, $d = d(OB)$
	\vskip 0.1cm
	\textit{Chứng minh.} Gọi hai đầu của đường Poincaré $OB$ là $P$ và $Q$ thỏa mãn $Q - O - B - P$, thì $d = ln(OB, PQ)$.
	Cho hai giá trị qua hàm mũ cơ số $e$:
	$e^d = (OB, PQ) = OP/OQ$. $BQ/BP$ (tất cả độ dài đại số) $= BQ/BP$ (độ dài thường) $= (r + OB)/(r-OB)$.
	Biến đổi thì sẽ có được kết quả trên.
	Ta chú ý vào hệ quả của nó: $d(OX) = d(OY) \Leftrightarrow OX = OY$, theo nghĩa Euclid
	\vskip 0.1cm
	Góp những gì đã chỉ ra, ta có: $OB' = OY'$, $OC' = OZ'$, $B'OC' = Y'OZ'$
	\vskip 0.1cm
	Lúc này sẽ tồn tại một phép quay tâm $O$, hợp với một phép đối xứng qua một đường thẳng nào đó qua $O$ nếu cần thiết (các phép biến hình của hình học Euclid) biến $B'$ thành $Y'$, $C'$ thành $Z'$, và bởi các phép này bảo toàn góc và độ dài cạnh, đường $P$ qua $B'C'$ sẽ biến thành đường $P$ qua $Y'Z'$, cho ta hai tam giác $OB'C'$ và $OY'Z'$ bằng nhau, hoàn tất chứng minh.
	\vskip 0.1cm
	Đoán xem ý tưởng về việc dịch chuyển hai tam giác dần dần lên nhau trong không gian này là của ai nào? Chính là ``chứng minh'' định lý $4$ của Euclid đây mà!
	Thật ra kết quả ta vừa chứng minh còn chi tiết hơn phát biểu ban đầu. Cụ thể là:
	\vskip 0.1cm
	Hai tam giác bằng nhau kiểu Poincare $\Leftrightarrow$ có hữu hạn phép $P$--đối xứng nào đó để biến tam giác này thành tam giác kia.
	\vskip 0.1cm
	Điểm cuối cùng là tiên đề về tính liên tục. May cho chúng ta là có một kết quả tốt trong mô hình Poincare như sau:
	\vskip 0.1cm
	Định lý $3$: Một đường tròn kiểu Poincare chính là một đường tròn kiểu Euclid trong gamma, và ngược lại. Tuy nhiên tâm của đường tròn kiểu Poincare chỉ trùng với tâm đường tròn kiểu Euclid khi nó là $O$.
	\vskip 0.1cm
	\textit{Chứng minh.} Nhận định sau được chứng minh nhờ vào hệ quả từ bổ đề trong chứng minh trường hợp c.g.c kiểu Poincare phía trên. 
	Xét trường hợp đường tròn Poincare delta có tâm $P$ là $O'$ khác $O$, sẽ tồn tại phép $P$--đối xứng $f$ biến $O'$ thành $O$. Bởi phép $P$--đối xứng bảo toàn độ dài kiểu Poincare, f(delta) sẽ là một đường tròn $P$ có tâm là $O$, như vừa chỉ phía trên, đây chính là một đường tròn kiểu Euclid tâm O. Tác động lần nữa thì f(f(delta)) = delta chính là một đường tròn Euclid, tuy nhiên có tâm khác $O'$.
	\vskip 0.1cm
	Đảo lại với một đường tròn Euclid delta nào đó, gọi $A$, $B$ là hai giao điểm của $OO'$ với delta, và $M$ là trung điểm của $AB$ kiểu Poincare. Khi ấy phép $P$--đối xứng biến $M$ thành $O$ có vai trò tương tự như $f$ ở trường hợp trên. Thậm chí ta chứng minh được $M$ chính là tâm Poincare của delta.
	\vskip 0.1cm
	Hệ quả: Nguyên lý liên tục đường tròn -- đường tròn đúng trong mô hình Poincaré.
	\vskip 0.1cm
	Tóm lại, qua những phát triển không dễ chịu lắm vừa rồi, chúng ta đã dịch được các tiên đề của hình hyperbolic thành định lý của hình Euclid trong mô hình Poincaré. Điều đó là đủ để khẳng định: nếu hình học phẳng của Euclid không chứa mâu thuẫn, thì hình học hyperbolic cho mặt phẳng cũng vậy.
	\vskip 0.1cm	
	$\pmb{11.}$ \textbf{\color{lichsutoanhoc}Hình học elliptic khác hình học Euclid như thế nào?}
	\vskip 0.1cm
	Nếu ta tiếp cận hình elliptic theo hướng về độ cong như của Riemann, thì ta có thể tìm ra một không gian nơi mà giả thuyết góc tù của Saccheri đúng. Theo cách nhìn về thiết lập tiên đề như những nhà Toán học đã được đề cập phía trên, ta sẽ không thể thay thế được Tiên đề $5$ bằng ``mọi đường thẳng đều cắt nhau'' ngay. Bởi kết hợp định lý $16$ và $27$ thì ta thấy là đường song song tồn tại. Mặt khác, nhìn vào mô hình của mặt phẳng elliptic ta dựng ở phần trước đây (bán cầu với các điểm đối xứng nhau qua tâm tính là $1$), việc một điểm nằm giữa hai điểm khác trông hơi sai sai, vì ``đường thẳng'' ở mô hình này trông như những đường tròn. Vì thế các tiên đề về sự nằm giữa sẽ được thay thế bằng các tiên đề về sự phân cách (separation axiom). Quan hệ về sự phân cách của $4$ điểm được ký hiệu là $(A, B | C, D)$, một các trực quan thì hình vẽ sau mô tả quan hệ ấy:
	\begin{figure}[H]
		\vspace*{-5pt}
		\centering
		\captionsetup{labelformat= empty, justification=centering}
		\includegraphics[width= 1\linewidth]{Vị trí điểm trong hình elliptic.pdf}
%		\caption{\small\textit{\color{}}}
		\vspace*{-10pt}
	\end{figure}
	Các tiên đề cụ thể là:
	Nếu $(A, B | C, D)$ thì $4$ điểm này thẳng hàng và phân biệt.
	\vskip 0.1cm
	Nếu $(A, B | C, D)$ thì $(C, D | A, B)$ và $(B, A | C, D)$.
	\vskip 0.1cm
	Nếu $(A, B | C, D)$ đúng thì $(A, C | B, D)$ sai.
	\vskip 0.1cm
	Chỉ một trong ba trường hợp sau xảy ra với $4$ điểm $A$, $B$, $C$, $D$ thẳng hàng và phân biệt: $(A, B | C, D)$, $(A, C | B, D)$ hoặc $(A, D | B, C)$
	\vskip 0.1cm
	Cho $3$ điểm $A$, $B$, $C$ thẳng hàng và phân biệt, khi ấy có điểm $D$ để $(A, B | C, D)$.
	\vskip 0.1cm
	Với $5$ điểm $A, B, C, D, E$ thẳng hàng và phân biệt, nếu đã sẵn $(A, B | D, E)$ thì hoặc $(A, B | C, D)$ hoặc $(A, B | C, E)$
	Phép chiếu xuyên tâm bảo toàn quan hệ này.
	\vskip 0.1cm
	Bởi những tiên đề về quan hệ nằm giữa đã bị thay thành tiên đề về sự phân tách, chứng minh cho định lý góc ngoài (định lý $16$ trong Elements) của ta không còn hợp lệ trong hình học elliptic, và các  tiên đề lúc này là không đủ để chỉ ra sự tồn tại của hai đường thẳng song song. 
	\vskip 0.1cm
	Chúng tôi cũng đã lý giải khái niệm về khoảng cách trong hình học này qua ở phần của Riemann, và đưa ra một mô hình cho hình học elliptic ở phần ngay sau đó: một nửa của mặt cầu, với các đường thẳng là các hình bán nguyệt nằm trên nửa mặt cầu đó, có tâm là tâm mặt cầu, và hai điểm nằm trên xích đạo được coi là một. 
	\vskip 0.1cm
	Một số điểm khác biệt so với hình học Euclid của hình học elliptic cũng tương tự như với hình học hyperbolic. Trong hình học elliptic, diện tích của một tam giác sẽ bị chặn và tỷ lệ thuận với hiệu của tổng ba góc tam giác đó (vẫn theo radian) với $\pi$, hai góc ở đỉnh của tứ giác Saccheri và góc thứ tư của tứ giác Lambert đều là góc tù. Không còn những tam giác đồng dạng nhưng không bằng nhau, vì thế hàm lượng giác cứ phải định nghĩa bằng chuỗi lũy thừa. Tất nhiên đây sẽ còn là một thế giới xa lạ với ta hơn cả hình hyperbolic, bởi cả quan hệ nằm giữa đã cần sửa thành sự phân tách để các tiên đề không dẫm vào chân nhau. 
	\vskip 0.1cm
	$\pmb{12.}$ \textbf{\color{lichsutoanhoc}Ý nghĩa và ứng dụng}
	\vskip 0.1cm
	Những ý tưởng mà Bolyai và Lobachevsky đã thúc đẩy là khá mới mẻ với giới tri thức đương thời. Dù vậy, thời gian đã dần xua tan những nghi ngờ còn sót lại của đại đa số học giả. Người ta dần học được vài điều. Những tiên đề không phải là chân lý hay sự thật vĩnh cửu, chúng chỉ là những quy ước con người đề ra. Khi thay đổi các tiên đề thì ta sẽ tạo ra các hệ thống kết quả khác nhau, chúng ta sẽ không nghi ngờ liệu nó có đúng trong thực tế hay không, mà sẽ đi vào phát triển những hệ thống cho ra những kết quả thú vị, hoặc để tìm ra một mâu thuẫn nào đó để bác bỏ hệ thống đó.
	\vskip 0.1cm
	$60$ năm sau bài nói chuyện vượt thời gian tại Gottingen, một lý thuyết mới đã làm khuynh đảo Vật Lý đương thời -- Thuyết Tương đối của Einstein. Thuyết Tương đối rộng còn chỉ ra rằng thời gian và không gian có thể bị cong dựa vào khối lượng vật chất tại một vị trí. Vì vậy, hình học của lý thuyết ấy, hóa ra, lại chẳng phải hình Euclid, mà là hình học mà những Bolyai, Lobachevsky và Riemann phát triển. Một cách kiểm chứng rực rỡ cho những gì những con người tiên phong ấy đã viết nên.
	\vskip 0.1cm
	\textbf{\color{lichsutoanhoc}\color{lichsutoanhoc}Tài liệu tham khảo}
	\vskip 0.1cm
	-- Wikipedia tiếng Anh.
	\vskip 0.1cm
	-- Euclidean and Non--Euclidean Geometries -- Development and History (Marvin Jay Greenberg).
	\vskip 0.1cm
	-- Men of Mathematics The Lives and Achievements of the Great Mathematicians from Zeno to Poincaré (Eric Temple Bell).
	\vskip 0.1cm
	-- The History of Mathematics An Introduction (David M. Burton).
	\vskip 0.1cm
	-- https: //www.youtube.com/playlist?list=PL\\jLK2cYtt-VBSBtvfhxx-DW3Zw3nOQHVZ
	\vskip 0.1cm
	-- https: //rosetta.vn/lequanganh/wp-conten\\t/
	uploads/sites/7/2018/07/Riemann.pdf
	\vskip 0.1cm
	-- https: //www.cut-the-knot.org/triangle/py\\
	thpar/Drama.shtml
\end{multicols}


%	Một trong những thành tựu lớn của nhà tiên phong thông thái David Hilbert $(1862-1943)$ là luận án Grundlagen der Geometrie (Nền tảng của hình học) nhằm thiết lập nền tảng logic vững chãi cho bộ môn lâu đời này. Trong đó ông đưa ra $21$ tiên đề và $6$ khái niệm nguyên thủy (các thuật ngữ không được định nghĩa), nhiều hơn hẳn so với bộ $5$ tiên đề và $0$ định nghĩa nguyên thủy của Euclid.
%	\vskip 0.1cm
%	Sở dĩ cần khái niệm nguyên thủy là bởi: để làm Toán, ta cần các đối tượng, khái niệm. Để biết ý nghĩa một đối tượng, ta sẽ cần định nghĩa chúng qua các đối tượng khác, nhưng quá trình ấy không thể tiếp tục mãi với ngày càng nhiều đối tượng cần định nghĩa. Do đó một số khái niệm sẽ không có định nghĩa. Các tiên đề được lập ra là để quy định cách hoạt động của các đối tượng không có định nghĩa này. Một trong các sai lầm Euclid mắc phải chính là cố định nghĩa hết các khái niệm cơ bản như đường thẳng và điểm.
%	\vskip 0.1cm
%	
%	Cần nói trước là không phải chỉ có một cách đúng để phát triển tiên đề cho hình học Euclid. Ta còn có các đóng góp đến từ G. Peano, M. Pieri, G. Veronese, O. Veblen, G. de B. Robinson, E. V. Huntington, H. G. Forder, và G. Birkhoff. Chẳng hạn như cách lập tiên đề của Pieri chỉ có hai khái niệm nguyên thủy là ``điểm'' và ``chuyển động''. Cách tiếp cận của Hilbert nổi tiếng hơn cả do sự thanh thoát và tương đồng với cách phát triển của Euclid
%	\vskip 0.1cm
%	Dựa trên công trình của Hilbert, các tiên đề cho hình học của Euclid có thể được chia ra thành: quan hệ ``nằm trên/ đi qua'' (incidence, viết tắt là NTĐQ), quan hệ ``nằm giữa'' của các điểm (betweenness), quan hệ ``bằng nhau'' (congruence), tính ``liên tục'' (continuity), và quan hệ song song (parallelism). 
%	\vskip 0.1cm




	
	
	\vskip 0.1cm
	Để cho vui, ta có một ví dụ như sau về việc mở rộng một mặt phẳng có tính chất Euclid lên mặt phẳng xạ ảnh.
	\begin{figure}[H]
		\vspace*{-5pt}
		\centering
		\captionsetup{labelformat= empty, justification=centering}
		\includegraphics[width= 0.85\linewidth]{Mặt phẳng xạ ảnh hữu hạn 1.pdf}
%		\caption{\small\textit{\color{}}}
		\vspace*{-15pt}
	\end{figure}
%	\begin{figure}[H]
%		\vspace*{-5pt}
%		\centering
%		\captionsetup{labelformat= empty, justification=centering}
%		\begin{tikzpicture}[line cap=round,line join=round,>=triangle 45,x=1.0cm,y=1.0cm]
%			\clip(-2.88,-1.3) rectangle (4.96,5.02);
%			\draw [line width=2.pt] (-1.,4.)-- (-2.,0.);
%			\draw [line width=2.pt] (-2.,0.)-- (4.,0.);
%			\draw [line width=2.pt] (4.,0.)-- (-1.,4.);
%			\draw [line width=2.pt] (-1.,4.)-- (-0.14000930590759372,1.4522368500791227);
%			\draw [line width=2.pt] (-0.14000930590759372,1.4522368500791227)-- (-2.,0.);
%			\draw [line width=2.pt] (-0.14000930590759372,1.4522368500791227)-- (4.,0.);
%			\begin{scriptsize}
%				\draw [fill=ududff] (-1.,4.) circle (2.5pt);
%				\draw[color=ududff] (-1.04,4.47) node {$A$};
%				\draw [fill=ududff] (-2.,0.) circle (2.5pt);
%				\draw[color=ududff] (-2.3,-0.27) node {$B$};
%				\draw [fill=ududff] (4.,0.) circle (2.5pt);
%				\draw[color=ududff] (4.26,-0.23) node {$C$};
%				\draw [fill=uuuuuu] (-0.14000930590759372,1.4522368500791227) circle (2.0pt);
%				\draw[color=uuuuuu] (0.06,1.79) node {$Ge$};
%			\end{scriptsize}
%		\end{tikzpicture}
%		\vspace*{-10pt}
%	\end{figure}
	Ở đây $4$ điểm $A$, $B$, $C$, $Ge$ và các cạnh nối chúng tạo thành mặt phẳng có tính chất Euclid, đơn giản là trường hợp $4$ điểm ở ví dụ phía trên. 
	Như ta có thể thấy, hai đường bất kì của mặt phẳng này, ví dụ như $AGe$ và $BC$, hay $AC$ và $BGe$, đều song song nhau, do chúng không có điểm chung. 
	\begin{figure}[H]
		\vspace*{-5pt}
		\centering
		\captionsetup{labelformat= empty, justification=centering}
		\includegraphics[width= 1\linewidth]{Mặt phẳng xạ ảnh hữu hạn 2.pdf}
%		\caption{\small\textit{\color{}}}
		\vspace*{-10pt}
	\end{figure}
	Ta thêm vào các điểm ở ``vô cùng'' là $D$, $E$, $F$, và một đường đi qua các điểm ở ``vô cùng'', là đường tròn đi qua ba điểm $D$, $E$, $F$. Khi đó, ta thu được một mặt phẳng xạ ảnh. Lúc này mọi cặp đường đều có giao điểm. Chẳng hạn như $AC$và $BGe$ sẽ có giao điểm $E$, còn $CGe$ và $(DEF)$ có giao điểm là $F$.
	\vskip 0.1cm
	Sau đây là một cách đặt tiên đề về các mục còn lại: 
	\vskip 0.1cm
	Quan hệ Nằm giữa: với $3$ điểm $X, Y, Z$ thì ``$Y$ nằm giữa $X$ và $Z$'' được ký hiệu là $X-Y-Z$.
	\vskip 0.1cm
	$1.$ Nếu $A - B - C$, ba điểm $A$, $B$, $C$ thẳng hàng và $C - B - A$.
	\vskip 0.1cm
	$2.$ Với mỗi hai điểm $B$, $D$ phân biệt, tồn tại $3$ điểm $A, C, E$ nằm trên đường thẳng $BD$ sao cho: $A - B - D$, $B - C - D$, $C - D - E$ 
	\vskip 0.1cm
	$3.$ Giữa ba điểm phân biệt thẳng hàng, chỉ có đúng một điểm nằm giữa hai điểm còn lại.
	\vskip 0.1cm
	$4.$ Với mỗi đường thẳng $l$ và $3$ điểm $A$, $B$, $C$ không nằm trên đó:
	\vskip 0.1cm
	-- Nếu $A$ và $B$, $A$ và $C$ nằm về cùng nửa mặt phẳng bờ $l$, $B$ và $C$ nằm về cùng một phía nửa mặt phẳng bờ $l$
	\vskip 0.1cm
	-- Nếu $A$ và $B$ không nằm cùng nửa mặt phẳng bờ $l$, $B$ và $C$ không nằm cùng nửa mặt phẳng bờ $l$, thì $A$ và $C$ nằm về cùng nửa mặt phẳng bờ $l$.
	\vskip 0.1cm
	Quan hệ bằng nhau: 
	\vskip 0.1cm
	C$1$. Cho hai điểm $A$, $B$ phân biệt. Khi ấy với mọi điểm $A'$ và tia $r$ có gốc $A'$, tồn tại $B'$ trên $r$ để $A'B' = AB$ 
	\vskip 0.1cm
	C$2$. $AB = CD$, $AB = EF$ thì $CD = EF$. Hơn nữa $AB = AB$
	\vskip 0.1cm
	C$3$. Nếu $A - B - C$, $A' - B' - C'$, $AB = A'B'$, $BC = B'C'$, thì $AC = A'C'$.
	\vskip 0.1cm
	C$4$. Cho góc $BAC$ khác góc bẹt và một tia $A'B'$ từ điểm $A'$, khi đấy có đúng một tia $A'x$ trên một nửa mặt phẳng bờ $A'B'$ nào đó để $BAC = B'A'x$.
	\vskip 0.1cm
	C$5$. $ \angle A = \angle B, \angle A = \angle C$ thì $ \angle A = \angle C$. Hơn nữa $ \angle A = \angle A$ 
	\vskip 0.1cm
	C$6$. (Trường hợp bằng nhau cạnh--góc--cạnh (c.g.c) của tam giác) Nếu hai tam giác có một cặp cạnh và góc xen giữa hai cạnh đó tương ứng bằng nhau, thì hai tam giác ấy bằng nhau
	\vskip 0.1cm
	Tính liên tục
	\vskip 0.1cm
	$1.$ Tiên đề Dedekind, Tiên đề Archimedes, Tiên đề Aristotle
	\vskip 0.1cm
	$2.$ Nguyên lý liên tục của đường tròn: Nếu đường tròn $(O)$ có một điểm nằm trong và một điểm nằm ngoài một đường tròn $(O')$ thì hai đường tròn có hai giao điểm.
	\vskip 0.1cm
	$3.$ Nguyên lý liên tục đường thẳng -- đường tròn (hệ quả của tiên đề trước): Nếu một đường thẳng đi qua một điểm nằm bên trong đường tròn thì nó cắt đường tròn tại hai điểm.
	\vskip 0.1cm
	Tiên đề song song
	\vskip 0.1cm
	Tiên đề số $5$ của Euclid.
	\vskip 0.1cm
	Tiên đề song song của hình học hyperbolic. 
	\vskip 0.1cm
	Tiên đề song song của hình học elliptic.
	\vskip 0.1cm
	$\pmb{5.}$ \textbf{\color{lichsutoanhoc}Hình học hyperbolic khác hình học Euclid như thế nào?}
	\vskip 0.1cm
	Như đã hứa, công chuẩn bị của chúng ta từ đầu giờ đây sẽ có ích. Khi mà tiên đề song song đã không còn đúng, những mệnh đề tương đương với nó cũng vậy, từ ấy ta có những kết quả đáng chú ý như sau:
	\vskip 0.1cm
	-- Góc ở đỉnh của tứ giác Saccheri và góc thứ tư của tứ giác Lambert không phải góc vuông, mà là góc nhọn. Thậm chí hình chữ nhật còn không tồn tại trong thế giới này. 
	\vskip 0.1cm
	-- Diện tích các tam giác bị chặn. Cả Bolyai và Lobachevsky đều phát hiện ra rằng diện tích tam giác trong hình học này tỷ lệ thuận với độ lệch -- hiệu của pi với tổng ba góc trong tam giác đó tính theo radian.
	\vskip 0.1cm
	-- Không tồn tại hai tam giác đồng dạng nhưng không bằng nhau nữa. Nói cách khác, hai tam giác đồng dạng thì bằng nhau. 
	\vskip 0.1cm
	Một hệ quả của điều này là chúng ta không thể định nghĩa các hàm lượng giác như trong hình học Euclid. Các hàm sin, cos quen thuộc giờ được định nghĩa thông qua giới hạn của chuỗi lũy thừa. Nhờ vào khai triển Maclaurin, ta có được kết quả sau trong hình Euclid: (góc đo bằng radian)
	\begin{align*}
		&\cos(x) =  1 - \frac{x^2}{2!} + \frac{x^4}{4!} - ... = \sum_{n=0}^{\infty} \frac{(-1)^n.x^{2n}}{(2n)!} \\
		&\sin(x) =  x - \frac{x^3}{3!} + \frac{x^5}{5!} - ... = \sum_{n=0}^{\infty} \frac{(-1)^n.x^{2n+1}}{(2n+1)!}
	\end{align*}
	Ta sẽ định nghĩa sin và cos trong hình hyperbolic như trên, tan được định nghĩa là bằng sin/cos.
	Đồng thời có hai hàm hyperbolic hữu ích mà ng ta cũng nghiên cứu ở đây là sinh và cosh, có chuỗi lũy thừa giống của sin và cos bỏ đi $(-1)^n$:
	\begin{align*}
		&\cosh(x) =  1 + \frac{x^2}{2!} + \frac{x^4}{4!} + ... = \sum_{n=0}^{\infty} \frac{x^{2n}}{(2n)!} \\
		&\sinh(x) =  x + \frac{x^3}{3!} + \frac{x^4}{4!} - ... = \sum_{n=0}^{\infty} \frac{x^{2n+1}}{(2n+1)!}
	\end{align*}
	-- Có hai đường gọi là đường song song giới hạn (limiting parallel). Cụ thể, với điểm $P$ và đường thẳng $l$ không đi qua nó, sẽ tồn tại hai đường thẳng $d_1, d_2$ đối xứng nhau qua đường vuông góc $d$ hạ từ $P$ xuống $l$. Hai đường này là hai đường tạo với $d$ góc nhỏ nhất mà song song với $l$. 
	\vskip 0.1cm
	-- Hằng số $\pi$  không thể được định nghĩa như ở hình Euclid, bởi chu vi của đường tròn chia cho bán kính của nó trong không gian này còn chẳng phải một tỷ số cố định. Những điều trông rất hợp lý trong hình học Euclid, chẳng hạn như tỷ lệ giữa chu vi và bán kính của mọi đường tròn là như nhau, lại không đúng ở đây. Trong không gian hình học hyperbolic độ cong $-1$, chu vi của đường tròn bán kính $r$ được tính theo công thức: $C = 2\pi.\sinh r$. 
	\vskip 0.1cm
	Để minh họa cho một số tính chất khác biệt trên, hãy cùng điểm qua mô hình Beltrami -- Klein và Poincaré.  
	\vskip 0.1cm
	$\pmb{6. }$ \textbf{\color{lichsutoanhoc}Mô hình Beltrami-Klein}
	\vskip 0.1cm
	Chúng tôi sẽ không trình bày y hệt như công trình gốc của Klein để đỡ phức tạp, nhưng những ý tưởng chủ đạo vẫn là do Klein cải thiện từ luận án của Beltrami.
	Cũng giống với hình học Euclid, hình học hyperbolic cũng có mặt phẳng xạ ảnh để hoàn thiện cho không gian hyperbolic thông thường. Dựa vào đó, Klein đã xây dựng mô hình dưới đây từ mặt phẳng đó để nghiên cứu rõ hơn.
	Xét một đường tròn $\gamma$ với tâm O và đường kính $OR$ trên mặt phẳng Euclid, gọi phần bên trong của $ \gamma$ là tập hợp các điểm $X$ sao cho $OX < OR$ (tức là không bao gồm điểm nằm trên đường tròn). Một dây cung của đường tròn là một đoạn nối hai điểm trên đường tròn đó.
	Trong mô hình Beltrami -- Klein, tập hợp các điểm sẽ là phần bên trong của đường tròn, và với mỗi hai điểm $A, B$ từ tập hợp đó, ta định nghĩa ``đường thẳng'' nối $A, B$ là dây cung của $ \gamma$ đi qua $AB$ và bỏ đi hai đầu mút. Ta gọi một dây cung nhu vậy là dây mở và ký hiệu là $A$) ($B$. Quan hệ ``nằm trên'' được định nghĩa giống hệt với trong hình học Euclid: $P$ nằm trên $A$) ($B$ nếu $P$ nằm trên đường thẳng $AB$ theo nghĩa hình Euclid và $P$ nằm giữa $A$ và $B$. Quan hệ ``nằm giữa'' với hình hyperbolic ở đây là y như quan hệ nằm giữa trong hình Euclid. Quan hệ ``bằng nhau'' sẽ được định nghĩa ở phần tới, do nó phức tạp hơn.
	Hình vẽ dưới đây là một minh họa để thấy tiên đề song song của hình hyperbolic đúng trong mô hình này. Nếu bạn thấy nó quá hiển nhiên và tự hỏi tại sao điều này không được phát hiện sớm hơn, nhớ rằng việc xét mô hình này đã là một ý tưởng đột phá rồi. 
	\begin{figure}[H]
		\vspace*{-5pt}
		\centering
		\captionsetup{labelformat= empty, justification=centering}
		\includegraphics[width= 0.9\linewidth]{Hơn một đường song song.pdf}
%		\caption{\small\textit{\color{}}}
		\vspace*{-10pt}
	\end{figure}
	Để chứng minh tính phi mâu thuẫn của hình hyperbolic tương đối với hình Euclid, đầu tiên ta chuẩn bị một ``cuốn từ điển'' để dịch các khái niệm nguyên thủy (``điểm'', ``đường'', ``nằm trên'', ``nằm giữa'' và ``bằng nhau'') trong hình hyperbolic sang một cách diễn giải trong hình Euclid, như cách mà ta đã làm với bốn khái niệm đầu tiên. Các đối tượng có định nghĩa cũng sẽ được dịch sang hình Euclid một cách tương tự. Ví dụ như định nghĩa cho quan hệ ``song song'' trong hình hyperbolic sẽ được hiểu bằng cách thay mọi từ ``đường thẳng'' trong phát biểu thành từ ``dây mở''. Sau khi các khái niệm được chuyển đổi xong, ta có thể chuyển sang cắt nghĩa các tiên đề. Chẳng hạn như một tiên đề NTĐQ được dịch về mô hình Beltrami -- Klein như sau:
	\vskip 0.1cm
	Tiên đề NTĐQ $1$ (Klein): Với mỗi hai điểm $A$, $B$ phân biệt ở phần bên trong đường tròn $\gamma$, tồn tại đúng một dây mở $l$ của $\gamma$ sao cho cả $A$ và $B$ nằm trên $l$.
	\vskip 0.1cm
	Nhiệm vụ tiếp theo của ta là chứng minh đây đúng là một định lý trong hình Euclid, và làm điều này với tất cả các diễn giải tiên đề còn lại. Một khi các tiên đề hình hyperbolic trong mô hình Beltrami -- Klein đã trở thành định lý trong hình học Euclid, thì mọi chứng minh là hình học hyperbolic có mâu thuẫn sẽ có nghĩa là một tập hợp các mệnh đề đúng (chính là các tiên đề hình hyperbolic trong mô hình của ta) trong hình Euclid đã dẫn tới mâu thuẫn, từ đó cho thấy một mâu thuẫn trong hình học Euclid. Điều này không thể xảy ra nếu ta giả sử hình Euclid là phi mâu thuẫn.
	Quay trở lại, ta vẫn cần chứng minh tiên đề vừa phát biểu kia là một định lý trong hình Euclid.
	Thật vậy, gọi $l$ là đường thẳng theo nghĩa trong hình Euclid đi qua $A$ và $B$. Bởi $l$ có đi qua điểm nào đó nằm trong $\gamma$, nó sẽ có đúng hai giao điểm với $\gamma$, gọi là $C$ và $D$ (Nguyên lý liên tục đường thẳng -- đường tròn của hình Euclid). Khi ấy dây mở $CD$ chính là ``đường thẳng'' đi qua $A$ và $B$, và theo tiên đề NTĐQ của hình Euclid ta có sự tồn tại duy nhất của $CD$.
	\vskip 0.1cm
	Sự tồn tại của  đường song song giới hạn có thể được chỉ ra khá dễ dàng trong mô hình Beltrami--Klein. Nhìn hình ta thấy, hai dây mở $A$) ($E$ và $B$) ($F$ đi qua $P$ chính là hai đường song song giới hạn khi xét đường thẳng là dây mở $A$) ($B$ và điểm $P$ không nằm trên đó.
	\begin{figure}[H]
		\vspace*{-5pt}
		\centering
		\captionsetup{labelformat= empty, justification=centering}
		\includegraphics[width= 0.9\linewidth]{Đường song song giới hạn Klein.pdf}
%		\caption{\small\textit{\color{}}}
		\vspace*{-10pt}
	\end{figure}
	Cuối cùng là dịch các tiên đề về bằng nhau sang mô hình Klein. Hướng quen thuộc sẽ là tạo ra một hệ thống độ đo góc và độ dài cạnh cho nó, tuy nhiên ta không thể diễn giải trực tiếp theo nghĩa hình Euclid. Theo tiên đề $2$ của Euclid thì có những đoạn độ dài lớn tùy ý, trong khi độ dài các dây cung của một đường tròn thì không vượt quá đường kính đường tròn đó.
	Thế nên ta tạm sang mô hình Poincaré trước, nơi ta định nghĩa sự bằng nhau của góc dễ hơn. 
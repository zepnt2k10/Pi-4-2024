\thispagestyle{cackithitoannone}
\pagestyle{cackithitoan}
\everymath{\color{cackithi}}
\graphicspath{{../cackithi/pic/}}
\blfootnote{\color{cackithi}$^1$Trường Phổ thông Năng khiếu, ĐHQG TpHCM.}
\begingroup
\AddToShipoutPicture*{\put(0,616){\includegraphics[width=19.3cm]{../bannercackithi}}}
\AddToShipoutPicture*{\put(52,522){\includegraphics[scale=1]{../tieude3.pdf}}}
\centering
\endgroup
\vspace*{190pt}

\begin{multicols}{2}
	Kỳ thi chọn học sinh giỏi quốc gia năm học $2023-2024$ đã diễn ra trong hai ngày $5, 6/1/2024$. Bài viết giới thiệu đề thi môn Toán của kỳ thi năm nay cùng một số bình luận và nhận định của tác giả bài viết về đề thi này.
	\vskip 0.1cm
	Đề thi năm nay có ưu điểm là đặt các vấn đề thú vị, không theo lối mòn, đặc biệt là không đi theo một mô--típ cũ suốt hơn mười năm nay là đề thi lúc nào cũng có hai bài hình. Điều này sẽ giúp cho việc dạy và học đều hơn, hướng đến cơ bản hơn, tránh bắt tủ. Tuy vậy, đề thi cũng có những nhược điểm về cấu trúc, diễn đạt và sắp xếp làm giảm đi đáng kể những điểm tích cực. Trước khi đi đến những đánh giá chung, chúng ta đi chi tiết vào từng bài toán thi.
	\vskip 0.1cm
	\textbf{\color{cackithi}Bài $\pmb1$:} Với mỗi số thực $x$, ta gọi $\lfloor x\rfloor$ là số nguyên lớn nhất không vượt quá $x$.
	\vskip 0.1cm
	Cho dãy số $\{a_n\}_{n=1}^\infty$ xác định bởi: $a_n = \dfrac{1}{4^{\lfloor-\log_4 n\rfloor}}, \forall n \ge 1$. Đặt $b_n =  \dfrac{1}{n^2}\left( {\sum\limits_{k = 1}^n {{a_k}}  - \frac{1}{{{a_1} + {a_2}}}} \right), \forall n \ge 1$. 
	\vskip 0.1cm
	$a)$ Tìm một đa thức $P(x)$ với hệ số thực sao cho $b_n = P\left(\dfrac{a_n}{n}\right), \forall n \ge 1$.
	\vskip 0.1cm
	\columnbreak
	$b)$ Chứng minh rằng tồn tại một dãy số nguyên dương $\{n_k\}_{k=1}^\infty$ tăng thực sự sao cho
	\begin{align*}
		\mathop {\lim }\limits_{k \to \infty } {b_{{n_k}}} = \frac{{2024}}{{2025}}. 
	\end{align*}
	Bài toán này có định dạng khá lạ so với các bài toán dãy số trong các kỳ thi năm trước, một dãy số dưới dạng tổng đòi hỏi phần xử lý đại số trước. Mặc dù đã có hướng dẫn trước để đi đến công thức mấu chốt
	\begin{align*}
		b_n = - \dfrac{1}{5}\left(\dfrac{a_n}{n}\right)^2 + \dfrac{a_n}{n}
	\end{align*}
	nhưng đây vẫn là một ý không hề hiển nhiên. Ý thứ hai thì không khó đối với một học sinh (hoặc sinh viên) được học bài bản về giới hạn và về tập số thực, nhưng với các em học sinh chỉ vừa làm quen với giải tích thì quả là quá tầm. 
	\vskip 0.1cm
	\textbf{\color{cackithi}Bài $\pmb2$:} Tìm tất cả các đa thức $P(x), Q(x)$ với hệ số thực sao cho với mỗi số thực $a$ thì $P(a)$ là nghiệm của phương trình: $x^{2023} + Q(a)\cdot x^2 + (a^{2024} + a)x + a^3 + 2025a = 0$.
	\vskip 0.1cm
	Đây là một bài phương trình hàm đa thức nhẹ nhàng, chủ yếu dựa vào sự chia hết, đồng dư đa thức.
	\vskip 0.1cm
	\textbf{\color{cackithi}Bài $\pmb3$:} Cho $ABC$ là tam giác nhọn với tâm đường tròn ngoại tiếp $O$. Gọi $A'$ là tâm của đường tròn đi qua $C$ và tiếp xúc $AB$ tại $A$, gọi $B'$ là tâm của đường tròn đi qua $A$ và tiếp xúc $BC$ tại $B$, gọi $C'$ là tâm đường tròn đi qua $B$ và tiếp xúc $CA$ tại $C$.
	\vskip 0.1cm
	$a)$ Chứng minh rằng diện tích tam giác $A'B'C'$ lớn hơn hoặc bằng diện tích tam giác $ABC$.
	\vskip 0.1cm
	$b)$ Gọi $X,Y,Z$ lần lượt là hình chiếu vuông góc của $O$ lên các đường thẳng $A'B',B'C', C'A'$. Biết rằng đường tròn ngoại tiếp tam giác $XYZ$ lần lượt cắt lại các đường thẳng $A'B', B'C', C'A'$ tại các điểm $X',Y',Z' (X' \ne X, Y' \ne Y, Z' \ne Z)$. Chứng minh rằng các đường thẳng $AX', BY', CZ'$ đồng quy.
	\vskip 0.1cm
	Bài toán hình học này khai thác một cấu hình khá thú vị, lời giải chủ yếu sử dụng phép biến đổi góc, tam giác đồng dạng. Bài này có hai ý khá độc lập nhưng đều có thể sử dụng chung phép đồng dạng (vị tự quay) biến $ABC$ thành $A'B'C'$. Đây là năm thứ hai xuất hiện bất đẳng thức hình học và nội dung này vẫn tiếp tục gây khó khăn cho các thí sinh. 
	\vskip 0.1cm
	\textbf{\color{cackithi}Bài $\pmb4$:} Người ta xếp $k$ viên bi vào các ô của một bảng $2024 \times 2024$ ô vuông sao cho hai điều kiện sau được thỏa mãn: mỗi ô không có quá một viên bi và không có hai viên bi nào được xếp ở hai ô kề nhau (hai ô được gọi là kề nhau nếu chúng có chung một cạnh).
	\vskip 0.1cm
	$a)$ Cho $k = 2024$. Hãy chỉ ra một cách xếp thỏa mãn cả hai điều kiện trên mà khi chuyển bất kỳ viên bi đã được xếp nào sang một ô tùy ý kề với nó thì cách xếp mới không còn thỏa mãn cả hai điều kiện nêu trên.
	\vskip 0.1cm
	$b)$ Tìm giá trị $k$ lớn nhất sao cho với mọi cách xếp $k$ viên bi thỏa mãn hai điều kiện trên ta có thể chuyển một trong số các viên bi đã được xếp sang một ô kề với nó mà cách xếp mới vẫn không có hai viên bi nào được xếp ở hai ô kề nhau.
	\vskip 0.1cm
	Bài $4$ là một bài cực trị tổ hợp có câu $a$ khá nhẹ nhàng, nhưng câu $b$ là một ý khó, dù không sử dụng kiến thức gì cao siêu nhưng tìm được $k$ đã khó, chứng minh lại càng khó hơn (một cách tiếp cận tự nhiên cho các bài toán kiểu thế này là thay $2024$ bằng một số nhỏ hơn để khảo sát). 
	\vskip 0.1cm
	\textbf{\color{cackithi}Bài $\pmb5$:} Với mỗi đa thức $P(x)$, ta đặt 
	\begin{align*}
		P_1(x) &= P(x), \forall x\in \mathbb{R};\\
		P_2(x) &= P\left(P_1(x)\right), \forall x\in \mathbb{R};\\
		&\ldots\\
		P_{2024}(x)&= P\left(P_{2023}(x)\right), \forall x \in \mathbb{R}.
	\end{align*}
	Cho $a$ là số thực lớn hơn $2$. Tồn tại hay không một đa thức $P(x)$ với hệ số thực thỏa mãn điều kiện: với mỗi $t\in (-a;a)$, phương trình $P_{2024} (x) = t$ có đúng $2^{2024}$ nghiệm thực phân biệt?
	\vskip 0.1cm
	Bài toán này khai thác một chủ đề truyền thống về nghiệm của đa thức. Bài này là bài nhẹ nhàng nhất của ngày $2$, chỉ cần bình tĩnh một chút sẽ thấy số $2024$ không có ý nghĩa gì ở đây, có thể thay bằng $n$ và từ đó thì sẽ nghĩ đến quy nạp. Với $n = 1$, một cách tự nhiên sẽ dẫn đến đa thức có dạng $x^2 - c$, với $c$ là hằng số dương. Bài này khai thác một ý không mới, nhưng phù hợp để đặt ở vị trí\linebreak Bài $5$.
	\vskip 0.1cm  
	\textbf{\color{cackithi}Bài $\pmb6$:} Với mỗi số nguyên dương $n$, gọi $\tau(n)$ là số các ước nguyên dương của $n$.
	\vskip 0.1cm
	$a)$ Giải phương trình nghiệm nguyên dương $\tau(n) + 2023 = n$ với $n$ là ẩn số.
	\vskip 0.1cm
	$b)$ Chứng minh rằng tồn tại vô số số nguyên dương $k$ sao cho có đúng hai số nguyên dương $n$ thỏa mãn phương trình $\tau(kn) + 2023 = n$. 
	\vskip 0.1cm
	Bài số học này khai thác một số tính chất cơ bản của hàm $\tau(n)$ -- số các ước số của số nguyên dương $n$. Bài này về ý tưởng thì nhẹ nhàng, phù hợp với một bài toán thi olympic, nhưng tính toán và xét trường hợp hơi phức tạp, nhất là trong bối cảnh thí sinh không được sử dụng máy tính cầm tay.
	\vskip 0.1cm 
	\textbf{\color{cackithi}Bài $\pmb7$:} Trong không gian, cho đa diện lồi $D$ sao cho tại mỗi đỉnh của $D$ có đúng một số chẵn các cạnh chứa đỉnh đó. Chọn ra một mặt $F$ của $D$. Giả sử ta gán cho mỗi cạnh của $D$ một số nguyên dương sao cho điều kiện sau được thỏa mãn: với mỗi mặt (khác mặt $F$) của $D$, tổng các số được gán với các cạnh của mặt đó là một số nguyên dương chia hết cho $2024$. Chứng minh rằng tổng các số được gán với các cạnh của mặt $F$ cũng là một số nguyên dương chia hết cho $2024$.
	\vskip 0.1cm
	Đây là bài tổ hợp có mô hình lý thuyết đồ thị. Với các thí sinh được học bài bản về lý thuyết đồ thị (đến mức có thể chuyển bài toán đề bài về bài toán đồ thị phẳng) thì Bài $7$ này còn nhẹ nhàng hơn Bài $4$. Nhưng ngược lại, nếu không được trang bị tốt thì chắc là cắn bút, nhất là trong áp lực của phòng thi. 
	\vskip 0.1cm
	Để bạn đọc cảm nhận được rõ hơn về đề thi cùng hướng giải, chúng tôi sẽ đi chi tiết hơn vào các bước giải cho các bài toán thi.
	\vskip 0.1cm
	Ở bài $1$, sau một lúc để định thần ta sẽ thấy dãy $a_n$ sẽ bắt đầu bởi $1$ số $1$, rồi $3$ số $4$, rồi $12$ số $16$ ... do ${a_n} = \frac{1}{{{4^{[ - {{\log }_4}n]}}}}$ nên một cách tự nhiên, ta chọn $s$ sao cho $4^s < n \le 4^{s+1}$  thì $a_n = 4^{s+1}$. Từ đó sẽ tính được tổng 
	\begin{align*}
			&\sum\limits_{i = 1}^n {a_i}\\
				 = &1 \!+\! (4 \!-\! 1)4 \!+\! (16 \!-\! 4){4^2} \!+\! ... \!+\! (n \!-\! {4^s}){4^{s \!+\! 1}} \\
			= &1 \!+\! 12\cdot(4^0 \!+\! 4^2 \!+\! ... \!+\! 4^{2s \!-\! 2}) \!+\! (n \!-\! 4^s)4^{s \!+\! 1}\\
			= &1 + \frac{3(4^{2s} - 1)}{5} + (n - 4^s)4^{s + 1}.
	\end{align*}
	Thay $4^s$ bởi $\frac{a_n}{4}$ vào và chú ý là  $\frac{1}{{{a_1} + {a_2}}} = \frac{1}{5}$  ta sẽ biến đổi về được ${b_n} =  - \frac{1}{5}{\left( {\frac{{{a_n}}}{n}} \right)^2} + \frac{{{a_n}}}{n}$. Ở ý $b)$, ta chỉ cần gọi $\alpha$ là nghiệm dương của phương trình  $- \frac{{{x^2}}}{5} + x = \frac{{2024}}{{2025}}$ và chứng minh tồn tại dãy con của dãy $\frac{a_n}{n}$  dần về $\alpha$. Với ý sau, chỉ cần dùng tính chất cơ bản sau đây về xấp xỉ:  $\lim \frac{{\lfloor{4^n}\alpha \rfloor}}{{{4^n}}} = \alpha$. 
	\vskip 0.1cm
	Ở bài $2$, từ điều kiện đề bài ta suy ra  $P(x)(P(x)^{2022} + Q(x)P(x) + x^{2024} + x) = - x(x^2+2025)$.
	\vskip 0.1cm
	Từ đây $P(x)$ là ước của $x(x^2+2025)$. Nếu $P(x)$ chia hết cho $x$ thì vế trái chia hết cho $x^2$, còn vế phải thì không, mâu thuẫn. Nếu $P(x) = k(x^2+2025)$ thì thay vào, ta suy ra $k(x^{2024}+x) + x$ chia hết cho $x^2 + 2025$. Sử dụng sự kiện $x^{2024} \equiv 2025^{1012} (\mod x^2 + 2025)$ ta sẽ suy ra ngay mâu thuẫn. Vậy chỉ còn lại trường hợp $P(x) = k$ với $k$ là hằng số, thay vào tính được $Q(x)$ tương ứng. 
	\vskip 0.1cm
	Ở bài $3$, ở câu $a)$ ta có thể chứng minh $\Delta CA'O \sim \Delta CAB$, suy ra $\frac{{{S_{COA'}}}}{{{S_{CAB}}}} = {\left( {\frac{{CO}}{{CB}}} \right)^2} = \frac{{{R^2}}}{{{a^2}}}$, từ đó  ${S_{C'OA'}} = {S_{COA'}} = \frac{{{R^2}}}{{{a^2}}} \cdot {S_{ABC}}$. Cùng các hệ thức tương tự, suy ra ${S_{A'B'C'}} = {S_{C'OA'}} + {S_{A'OB'}} + {S_{B'OC'}} = {S_{ABC}} \cdot {R^2}\left( {\frac{1}{{{a^2}}} + \frac{1}{{{b^2}}} + \frac{1}{{{c^2}}}} \right)$. Cuối cùng, sử dụng bất đẳng thức quen thuộc $a^2 + b^2 + c^2 \le 9R^2$ là hoàn tất chứng minh. Ở câu $b)$, ta chứng minh hai tam giác $ABC$  và $X'Y'Z'$  có các cạnh tương ứng song song bằng cách chẳng hạn $AB$  và $X'Y'$ cùng vuông góc với $OB'$. Khi đó $AX',BY',CZ'$  đồng quy tại tâm vị tự. 
	\vskip 0.1cm
	Với bài $4$, ta gọi một cách xếp $k$ viên bi là \textit{tốt} nếu nó thỏa mãn hai điều kiện đề bài và một cách xếp là \textit{cứng} nếu nó tốt nhưng nếu di chuyển bất kỳ một viên bi nào sang ô kề với nó thì cách xếp không còn tốt. Ý $a)$ là quá đơn giản vì ví dụ về đường chéo là rất hiển nhiên. Ở ý $b)$ điểm khó của bài toán nằm ở chỗ nếu có cách xếp cứng với $k$ viên bi thì chưa chắc đã có cách xếp cứng với $k+1$ viên bi. Ta cần tìm $k_0$ nhỏ nhất sao cho từ $k_0$  trở đi tồn tại cách xếp cứng. Được gợi ý bởi ý tưởng đường chéo, ta đưa ra thuật toán xây dựng cách xếp cứng cho mọi $k \ge 4046$ như sau. Đánh số các đường chéo song song với đường chéo chính là $D_1, D_2, \ldots, D_{4047}$. Đầu tiên ta đặt đầy bi vào hai đường chéo $D_{2022}$ và $D_{2026}$, cùng với hai viên bi ở đầu của đường chéo $D_{2024}$. Đây là một cách xếp cứng với $k = 4046$. Sau đó ta sẽ dùng các ô trên đường chéo chính $D_{2024}$ để tăng số bi dần dần (hành lang tạo bởi $D_{2022}$ và $D_{2026}$ sẽ đảm bảo tính cứng của cách xếp), khi đầy $D_{2024}$ rồi thì lại lấy bớt đi và thay bằng nguyên một đường chéo bên dưới (hay trên, cách $1$) đường chéo có bi. Như thế ta sẽ xây dựng được các cách xếp cứng cho mọi $k$ từ $4046$ cho đến $2024^2/2$. Với $k > 2024^2/2$ thì dễ dàng chứng minh sẽ phải có $2$ ô kề nhau, tức là không có cách xếp tốt. Để chứng minh $k = 4045$ là giá trị lớn nhất cần tìm gọi các ta tô màu các ô của hình vuông bằng hai màu đen trắng xen kẽ nhau. Ta lần lượt chứng minh các sự kiện sau $i)$ một cách xếp cứng sẽ chỉ có các viên bi trên các ô cùng một màu $ii)$ Nếu một cách xếp cứng có ít nhất $4045$ viên bi thì nó có ít nhất $4046$ viên bi. Ở ý $i)$ ta dùng phản chứng và nguyên lý cực hạn (xét hai đường chéo đen, trắng chứa bi gần nhau nhất), khi đó bi đứng đầu hàng của đường chéo ngắn hơn sẽ di chuyển được. Ở ý $ii)$, ta cũng sử dụng nguyên lý cực hạn, chọn ra đường chéo có chỉ số nhỏ nhất và lớn nhất có bi rồi chứng minh rằng toàn bộ các ô của $2$ đường chéo này đều có bi. Tiếp theo, ta phải cố gắng xử lý kỹ thuật, ``tránh" trường hợp toàn bộ đường chéo đều có bi ở câu $a)$. Câu $a)$ là một gợi ý quan trọng cho việc ``tránh" này. 
	\vskip 0.1cm
	Ở bài $5$, một cách tự nhiên ta nghĩ đến quy nạp. Điểm mấu chốt ở đây là ta phải làm mạnh mệnh đề lên thành $P_n(x)$ có $2^n$ nghiệm phân biệt thuộc $(-a, a)$ (lúc đó mới sử dụng giả thiết quy nạp được). Với $n = 1$, do tính đối xứng ta sẽ chọn $P(x) = P_1(x) = x^2 - c$. Để phương trình $P_1(x) = t$ có đúng $2$ nghiệm thực với mọi số thực $t \in (-a, a)$ ta sẽ cần có $c > a$ và để hai nghiệm này thuộc $(-a, a)$ ta cần có $c < a^2 - a$. Vì $a > 2$ nên chọn được $c$ như vậy. Đến đây thì mọi việc dễ dàng rồi:  $P_{k+1}(x) = 0 \Leftrightarrow P_k(P(x)) = 0$. Vì phương trình $P_k(x) = 0$ có $2^k$ nghiệm phân biệt thuộc $(-a, a)$ và ứng với mỗi nghiệm $x_i$ ấy phương trình $P(x) = x_i$ lại có $2$ nghiệm phân biệt dạng  $\pm\sqrt{x_i+c}$ nên dễ dàng suy ra phương trình $P_{k+1}(x)$ có $2^{k+1}$ nghiệm phân biệt.    
	\vskip 0.1cm
	Ở bài $6$, ý tưởng đầu tiên là sử dụng đánh giá $\tau (n) \le 2\sqrt n $  để chặn $n$. Bất đẳng thức này có thể chứng minh dễ dàng dựa vào nhận xét sơ đẳng: các ước số của $n$ chia thành các nhóm $(a, b)$ với $ab = n$. Từ đó áp dụng vào bài toán, ta suy ra được đánh giá $n < 2115$. Từ đây suy ra $2025 < n < 2115$ và như vậy $n$ không chính phương, suy ra $\tau(n)$ chẵn, suy ra $n$ lẻ. Dùng công thức tính $\tau(n) = (\alpha_1+1)\ldots(\alpha_r+1)$ với $n = p_1^{{\alpha _1}}...p_r^{{\alpha _r}}$  ta chứng minh được nếu $n$ lẻ và $2025 < n < 2115$ thì  $\tau(n) \le 18$  (chú ý do $3\cdot5\cdot7\cdot11\cdot13 > 2115$ nên $r \le 4$). Từ đây tiếp tục chặn được $n$ chặt hơn, thử các trường hợp ta đi đến kết luận không tồn tại $n$ sao cho  $\tau (n) + 2023 = n$. Với ý $b)$, ý tưởng cơ bản là chọn $k$ nguyên tố đủ lớn, khi đó thì $\tau (kn) \le \tau (k)\tau (n) \le 4\sqrt n$  từ đó cũng chặn được $n \le 2211$. Tương tự như ở trên, ta cũng đánh giá được $\tau(n) \le 18$. Lúc này do chọn $k$ nguyên tố lớn nên $(k, n) = 1$ và $\tau (kn) = \tau (k)\tau (n) = 2\tau (n)$. Vì $n$ không chính phương, $\tau(n)$ chẵn nên $n \equiv 3 \pmod 4$, từ đó thử từng trường hợp $n$ từ $2027$ đến $2059$ (cách $4$) ta tìm được đúng $2$ nghiệm là $2027$ và $2031$.
	\vskip 0.1cm 
	Ở bài $7$, đầu tiên ta chuyển mô hình đa diện thành mô hình đồ thị phẳng. Điều này có thể thực hiện bằng cách chiếu các đỉnh của đồ thị lên một mặt cầu nằm trong đa diện rồi dùng phép nghịch đảo có tâm $P$ (không trùng các đỉnh) nằm trên mặt cầu biến mặt cầu thành mặt phẳng. Sau đó ta xét đồ thị đối ngẫu của đồ thị phẳng, có đỉnh là các miền và hai miền có cạnh chung được sẽ được nối với nhau bởi một cạnh. Từ điều kiện đề bài suy ra đồ thị đối ngẫu cũng là đồ thị phẳng có các miền là các chu trình chẵn, từ đó chứng minh được mọi chu trình của nó đều chẵn, suy ra nó lưỡng phân, như vậy có thể tô màu các đỉnh bằng hai màu xanh, đỏ sao cho hai đỉnh kề nhau không cùng màu. Ngược trở lại với bài toán ban đầu, ta có thể tô màu các mặt của đa diện sao cho các mặt kề nhau không cùng màu. Đến đây thì mọi việc đơn giản. Giả sử $F$ được tô màu xanh. Xét $S_1$ là tổng các số viết trên các cạnh kề với mặt xanh khác $F$ và $S_2$ là tổng các số viết trên các cạnh kề với mặt đỏ. Vì mỗi cạnh không kề với $F$ sẽ xuất hiện trong cả hai tổng $S_1$ và $S_2$ đúng $1$ lần còn mỗi cạnh kề $F$ xuất hiện trong $S_2$ một lần nhưng không xuất hiện trong $F_1$. Từ đó tổng các số viết trên các cạnh kề với $F$ bằng $S_2 - S_1$. Vì mỗi cạnh chỉ kề với đúng một mặt xanh và một mặt đỏ nên $S_2$ chẳng qua là tổng các số trên các mặt đỏ và $S_1$ là tổng các số trên các mặt xanh khác $F$. Từ đây suy ra điều cần chứng minh.  
	\vskip 0.1cm
	Đề thi năm nay được đánh giá là khó, có nhiều bất ngờ gây ... sốc. Ở ngày thứ nhất, với bốn bài toán và bảy ý cần có sự sắp xếp và chọn lựa phù hợp hơn. Bài số $1$ nên là một bài nhẹ nhàng, chân phương thay vì những thách thức. Bài $1$ luôn là bài tạo tâm lý tốt (hay xấu) cho học sinh trong cả cuộc thi. Trước đây có một số năm cũng có Bài $1$ ``sát thủ" (như các năm $1997$, $2010$) và kết quả là năm đó học sinh làm bài kém hơn thường lệ, dù tổng thể đề không khó hơn. Vì ngày thứ nhất có đến bốn bài nên cũng nên cân nhắc về độ khó của các bài toán. So sánh giữa $5$ điểm của Bài $4$ với $6$ điểm của Bài $5$ thì thấy rõ sự khác biệt.
	\vskip 0.1cm
	Ngoài vấn đề về độ khó, đề thi năm nay còn có một số điểm thiếu hợp lý. Thứ nhất, đó là chọn chủ đề, hai bài toán đại số của đề thi đều thuộc chủ đề đa thức (đó là chưa kể Bài $1a$ cũng sử dụng từ đa thức). Đại số là phân môn có các chủ đề đa dạng nhất (đa thức, phương trình hàm, phương trình--hệ phương trình, bất đẳng thức, tổng và tích ...) và sở trường của các thí sinh ở mỗi phần cũng khác nhau nên đề thi không nên chọn trùng lặp. Thứ hai, trong đề thi năm nay, chủ đề tổ hợp xuất hiện trong hai bài. Điều này là tốt theo ý nghĩa sẽ khuyến khích học tổ hợp thay vì lâu nay luyện hình học hơi sâu. Nhưng vì hai bài này đều đặt ở vị trí cuối cùng của mỗi ngày nên chắc là sẽ rất ít học sinh đụng tới (vì về mặt tâm lý, gặp tổ hợp học sinh đã ngại, lại là bài cuối nên bỏ luôn). Đúng ra, nên chọn một bài dễ hơn (hoặc lấy Bài $7$ nhưng phát biểu chân phương hơn) để đặt lên vị trí đầu (Bài $1$ hoặc Bài $5$). Như năm $2012$ có hai bài tổ hợp thì Bài $5$ nhẹ nhàng và chân phương, mang tính khuyến khích rất cao.
	\vskip 0.1cm
	{\bf\color{cackithi} Thông tin cập nhật.} Sau khi bản thảo này đã được lên trang, Bộ Giáo dục và Đào tạo đã công bố kết quả kỳ thi học sinh giỏi Toán quốc gia. Kết quả đã phản ánh sát những lo ngại mà tác giả đã chỉ ra. Theo đó, có xấp xỉ $43 \%$ thí sinh đoạt giải, với các ngưỡng điểm đạt giải Nhất, Nhì, Ba và Khuyến Khích tương ứng là $22$, $16$, $11,5$ và $7$ (trên tổng số điểm tối đa là $40$). Đây là những con số thấp kỷ lục trong vòng nhiều năm trở lại đây. Với $31,5$ điểm, em Tạ Đức Anh, học sinh lớp $12$ trường THPT chuyên Đại học Sư phạm Hà Nội, là thủ khoa của kỳ thi. Ngoài ra, em Huỳnh Nguyên Phát, học sinh lớp $12$ trường THPT chuyên Chu Văn An, Bình Định, đã đem về giải Nhất đầu tiên của môn Toán cho tỉnh Bình Định. Bên cạnh đó, em Hà Nhật Minh, học sinh lớp $12$ trường THPT chuyên Lê Quý Đôn, Điện Biên cũng trở thành học sinh đầu tiên của tỉnh này đạt giải Nhì môn Toán, đồng thời lọt vào vòng thi chọn đội tuyển Toán tham dự kỳ thi IMO tới đây.
	\begin{figure}[H]
		\vspace*{-5pt}
		\centering
		\captionsetup{labelformat= empty, justification=centering}
		\includegraphics[height= 0.465\linewidth]{Ta_Duc_Anh}
		\includegraphics[height= 0.465\linewidth]{Huynh_Nguyen_Phat}
		\includegraphics[height= 0.465\linewidth]{Ha_Nhat_Minh}
		\caption{\small\textit{\color{cackithi}Từ trái sang phải: Tạ Đức Anh, Huỳnh Nguyên Phát và Hà Nhật Minh.}}
		\vspace*{-5pt}
	\end{figure}
\end{multicols}
\newpage
\begingroup
\AddToShipoutPicture*{\put(152,696){\includegraphics[scale=1]{../tieude1.pdf}}}
\centering
\endgroup
\vspace*{5pt}

\begin{multicols}{2}
	Trong phần đầu chuyên mục, chúng tôi sẽ trình bày với các bạn lời giải các bài toán trong kỳ thi Olympic toán vùng Trung Mỹ và Caribê năm $2023$ đăng trong số báo $11/2023$. 
	\begin{figure}[H]
		\vspace*{-5pt}
		\centering
		\captionsetup{labelformat= empty, justification=centering}
		\includegraphics[width= 0.85\linewidth]{gocolympic}
%		\caption{\small\textit{\color{}}}
		\vspace*{-15pt}
	\end{figure}
	{\bf\color{cackithi} OC$\pmb{55.}$} Tìm tất cả các cách tô màu các số nguyên dương sao cho điều kiện sau thỏa mãn:  
	\vskip 0.1cm
	$\bullet$ Mỗi số có màu xanh hoặc đỏ;
	\vskip 0.1cm
	$\bullet$  Tổng của hai số (không nhất thiết phân biệt) cùng màu bất kỳ có màu xanh.
	\vskip 0.1cm
	\textit{Lời giải.} Từ điều kiện bài toán ta suy ra tất cả các số chẵn có màu xanh. Hơn nữa nếu $2k+1$ là số lẻ đầu tiên có màu xanh thì tất cả các số lẻ tiếp theo đều có màu xanh do là tổng của 2 số màu xanh: 
	\begin{align*}
		2k+3&=(2k+1) + 2, 2k+5\\
		&=(2k+1) + 4, \cdots
	\end{align*}
	Như vậy cách tô màu thỏa mãn đầu bài có một trong hai dạng như sau:
	\vskip 0.1cm
	-- Tất cả các số chẵn tô màu xanh, tất cả các số lẻ tô màu đỏ.
	\vskip 0.1cm
	-- Các số lẻ $1, 3, \cdots, 2k-1 (k\ge 1)$ tô màu đỏ, tất cả các số còn lại tô màu xanh. 
	\vskip 0.1cm
	Có thể dễ dàng kiểm tra các cách tô này đều thỏa mãn điều kiện đầu bài.
	\vskip 0.1cm
	{\bf\color{cackithi} OC$\pmb{56.}$} Octavio viết một số nguyên dương $n$ lên bảng  và sau đó anh bắt đầu một quá trình trong đó, ở mỗi bước, anh xóa số nguyên $k$ được viết trên bảng  và thay thế nó bằng một trong các số sau:
	\begin{align*}
		3k-1, \quad 2k+1, \quad \frac{k}{2},
	\end{align*} 
	với điều kiện số mới viết là số nguyên.
	\vskip 0.1cm
	Chứng minh rằng với mọi số nguyên dương $n$, Octavio có thể viết lên bảng  số $3^{2023}$ sau hữu hạn bước.
	\vskip 0.1cm
	\textit{Lời giải.} Ta sẽ chứng minh bằng quy nạp theo $n.$ Ta có các nhận xét sau:
	\vskip 0.1cm
	Nhận xét $1$: Nếu $k$ lẻ, từ $k$ ta có thể nhận được $3k.$  Thật vậy 
	\begin{align*}
		 k\rightarrow 3k-1 \rightarrow \frac{3k-1}{2} \rightarrow 2\frac{3k-1}{2} +1=3k.
	\end{align*}
	Nhận xét $2$: Từ số $k$ ta có thể nhận được $3k+1.$  Thật vậy 
	\begin{align*}
		k&\rightarrow 2k+1 \rightarrow 3(2k+1)-1=6k+2 \\
		&\rightarrow \frac{6k+2}{2}=3k+1.
	\end{align*}
	Theo Nhận xét $1$, xuất phát từ số $1$ ta sẽ nhận được số $3^{2023},$ tức là khẳng định đúng với $n=1.$ Giả sử khẳng định đúng với mọi số $k< n,$ ta chứng minh nó cũng đúng với $n.$ 
	\vskip 0.1cm
	Nếu $n$ chẵn, ta viết được $\frac{n}{2}$ và áp dụng giả thiết quy nạp, ta có điều cần chứng minh. Với $n$ lẻ, ta viết được $3n-1$ và $3n+1$ (theo Nhận xét $2$). Chú ý rằng trong hai số chẵn liên tiếp  $3n-1$ và $3n+1$ phải có ít nhất một số chia hết cho $4,$ như vậy ta sẽ viết được một số nhỏ hơn $n$ là $\frac{3n-1}{4}$ hoặc $\frac{3n+1}{4}.$ Từ đó áp dụng quy nạp ta có điều phải chứng minh.
	\vskip 0.1cm
	{\bf\color{cackithi} OC$\pmb{57.}$} Trong một cái ao có $n (n \geq 3)$  hòn đá  xếp thành vòng tròn. Một công chúa muốn đánh số những hòn đá với các số $1, 2, \dots, n$ theo thứ tự nào đó rồi đặt một số con cóc lên những hòn đá. Sau khi đặt tất cả các con cóc vào vị trí, chúng bắt đầu nhảy theo quy tắc sau: khi một con cóc đến hòn đá có đánh số $k$, nó đợi $k$ phút rồi nhảy sang hòn đá liền kề theo chiều kim đồng hồ.
	\vskip 0.1cm
	Hỏi số lượng cóc nhiều nhất là bao nhiêu để công chúa có thể đánh số các hòn đá và đặt các con cóc sao cho không bao giờ có hai con cóc ở trên cùng một hòn đá trong thời gian từ một phút trở lên?
	\vskip 0.1cm
	\textit{Lời giải.}
	Ta nói hai con cóc gặp nhau nếu chúng ở trên cùng một hòn đá trong thời gian từ một phút trở lên. Ký hiệu $a_1, a_2, \dots, a_n$ là số của các hòn đá lần lượt theo chiều kim đồng hồ. Chúng ta coi các chỉ số modulo $n,$ tức là $a_{n+i}\equiv a_i.$ 
	\vskip 0.05cm
	Trước tiên ta chứng minh nhận xét sau: Giả sử có các con cóc trên hòn đá số $a_l$ và $a_k,$ khi đó chúng sẽ không bao giờ gặp nhau nếu và chỉ nếu $a_l+a_{l+1}+\dots+a_{k-1}\ge a_m $ với mọi $1\le m\le n.$
	\vskip 0.05cm
	{\it Chứng minh:} Giả sử hai con cóc không bao giờ gặp nhau. Nếu tồn tại $m$ để $a_l+a_{l+1}+\dots+a_{k-1}< a_m $ thì 
	\begin{align*}
		&a_l\!+\!a_{l\!+\!1}\!+\!\dots\!+\!a_{k\!-\!1}\!+\!a_k\!+\! a_{k\!+\!1}\!+\!\dots \!+\!a_{m\!-\!1} \\
		<\, &a_k\!+\! a_{k\!+\!1}\!+\!\dots \!+\!a_{m\!-\!1}\!+\!a_m.
	\end{align*} 
	Như vậy hai con cóc sẽ gặp nhau tại hòn đá $a_m$ sau $a_l+a_{l+1}+\dots+a_{m-1}$ phút và dẫn đến mâu thuẫn. 
	\vskip 0.05cm
	Chiều ngược lại giả sử $a_l+a_{l+1}+\dots+a_{k-1}\ge a_m $ với mọi $1\le m\le n.$ Nếu hai con cóc gặp nhau tại hòn đá $a_m$ thì ta có $a_l+\dots a_{k-1}+a_k+\dots +a_{m-1} < a_k+\dots +a_{m-1}+a_m$ và điều này dẫn đến mâu thuẫn. 
	\vskip 0.05cm
	Ta sẽ chứng minh số lượng cóc nhiều nhất có thể là $ \left\lceil \frac{n}{2} \right\rceil.$ Thật vậy, nếu công chúa đặt $ \left\lceil \frac{n}{2} \right\rceil+1$ con cóc. Với mỗi con cóc ở hòn đá $a_i$ ta xét hòn đá $a_{i+1}.$ Phải có ít nhất $2$ giá trị $i$ mà hòn đá $a_{i+1}$ cũng có cóc vì nếu trái lại thì có ít nhất $ \left\lceil \frac{n}{2} \right\rceil$ hòn đá trống và điều này mâu thuẫn vì dẫn đến số cóc bé hơn $\left\lceil \frac{n}{2} \right\rceil+1.$ Ta suy ra phải có số $a_i<n$ mà trên cả hai hòn đá $a_i$ và $a_{i+1}$ đều có cóc. Như vậy, theo nhận xét trên thì hai con cóc trên đó sẽ gặp nhau tại hòn đá có số $n,$ điều này mâu thuẫn với giả thiết.
	\vskip 0.1cm
	Bây giờ giả sử có $ \left\lceil \frac{n}{2} \right\rceil$ con cóc. Công chúa đánh số các hòn đá lần lượt theo chiều kim đồng hồ như sau:
	\begin{align*}
		n, 1, n-1, 2, n-2, \dots k, n-k, \dots.
	\end{align*}
	Nếu công chúa đặt các con cóc vào hòn đá số $n$ và các hòn số $k<\frac{n}{2}$ thì điều kiện trong nhận xét thỏa mãn và các con cóc không bao giờ gặp nhau. Như vậy số lượng cóc nhiều nhất có thể xếp là $ \left\lceil \frac{n}{2} \right\rceil.$
	\vskip 0.1cm
	Trong phần cuối của chuyên mục kỳ này, chúng tôi sẽ giới thiệu với bạn đọc các bài toán chọn lọc trong kỳ thi Olympic toán học trẻ của Hàn Quốc năm $2023$. Các bài toán này phù hợp với trình độ học sinh lớp $8-10$.
	\vskip 0.1cm
	{\bf\color{cackithi} OC$\pmb{64}$.} Tìm tất cả các cặp số nguyên $(x, y)$ thỏa mãn
	\begin{align*}
		y^2 = x^3 + 2x^2 + 2x + 1.
	\end{align*}
	{\bf\color{cackithi} OC$\pmb{65}$.} Cho $n (n\geq 5) $ là một số nguyên dương. Có $n$ viên đá trắng và $n$ viên đá đen (tổng cộng $2n$ viên) xếp thành một hàng trong đó $n$ viên đầu tiên có màu trắng và $n$ viên tiếp theo có màu đen.
	\begin{figure}[H]
		\vspace*{-5pt}
		\centering
		\captionsetup{labelformat= empty, justification=centering}
		\includegraphics[width= 0.7\linewidth]{OC65}
%		\caption{\small\textit{\color{}}}
		\vspace*{-10pt}
	\end{figure}
	Ta có thể thực hiện thao tác sau: chọn một số nguyên dương bất kỳ $k (k \leq 2n - 5)$ và  đổi chỗ viên đá ở vị trí thứ $k$ với viên đá ở vị trí thứ  $(k+5).$ 
	\vskip 0.1cm
	Tìm tất cả các số nguyên dương $n$ sao cho chúng ta có thể làm cho $n$ viên đá đầu tiên có màu đen và   $n$ viên đá tiếp theo có màu trắng sau một số hữu hạn lần thao tác.
	\vskip 0.1cm
	{\bf\color{cackithi} OC$\pmb{66}$.} Có $2023$ tay vợt  tham gia  một giải đấu quần vợt theo thể thức vòng tròn, hai tay vợt bất kỳ đấu với nhau đúng một trận. Biết rằng không có trận hòa và không có tay vợt nào thắng tất cả các tay vợt khác. Ta  gọi tay vợt $A$ là ``có kỹ năng" nếu với mỗi tay vợt $B$ thắng $A$, có một tay vợt $C$ thắng $B$ và thua $A$.
	\vskip 0.1cm
	Biết rằng có đúng $N (N\geq 0)$ tay vợt ``có kỹ năng". Tìm giá trị nhỏ nhất của $N$.	
\end{multicols}
